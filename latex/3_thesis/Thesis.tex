%% ----------------------------------------------------------------
%% Thesis.tex -- MAIN FILE (the one that you compile with LaTeX)
%% ---------------------------------------------------------------- 

% Set up the document
\documentclass[a4paper, 11pt, oneside]{Thesis}  % Use the "Thesis" style, based on the ECS Thesis style by Steve Gunn
\graphicspath{Figures/}  % Location of the graphics files (set up for graphics to be in PDF format)

% Include any extra LaTeX packages required
%\usepackage[square, numbers, comma, sort&compress]{natbib}  % Use the "Natbib" style for the references in the Bibliography
\usepackage{natbib}  % Use the "Natbib" style for the references in the Bibliography
\usepackage{verbatim}  % Needed for the "comment" environment to make LaTeX comments
\usepackage{vector}  % Allows "\bvec{}" and "\buvec{}" for "blackboard" style bold vectors in maths
\hypersetup{urlcolor=blue, colorlinks=true}  % Colours hyperlinks in blue, but this can be distracting if there are many links.


\usepackage[spanish]{babel}
\usepackage[utf8x]{inputenc}
\usepackage{enumerate}
\usepackage{float}
\usepackage{graphicx,subfigure}
%% ----------------------------------------------------------------
\begin{document}
	
	
\frontmatter      % Begin Roman style (i, ii, iii, iv...) page numbering



% Set up the Title Page
\title  {Detección \textit{in Silico} de Neoantígenos Utilizando \textit{Transformers} y \textit{Transfer Learning} en el Marco de Desarrollo de Vacunas Personalizadas para Tratar el Cáncer}
\authors  {\texorpdfstring
            {\href{vmachacaa@unsa.edu.pe}{Vicente Machaca Arceda}}
            {Author Name}
            }
\addresses  {\groupname\\\deptname\\\univname}  % Do not change this here, instead these must be set in the "Thesis.cls" file, please look through it instead
\date       {\today}
\subject    {}
\keywords   {}

\maketitle
%% ----------------------------------------------------------------

\setstretch{1.3}  % It is better to have smaller font and larger line spacing than the other way round

% Define the page headers using the FancyHdr package and set up for one-sided printing
\fancyhead{}  % Clears all page headers and footers
\rhead{\thepage}  % Sets the right side header to show the page number
\lhead{}  % Clears the left side page header





\pagestyle{fancy}  % Finally, use the "fancy" page style to implement the FancyHdr headers

%% ----------------------------------------------------------------
% Declaration Page required for the Thesis, your institution may give you a different text to place here
\Declaration{

\addtocontents{toc}{\vspace{1em}}  % Add a gap in the Contents, for aesthetics

I, Yo Vicente Machaca Arceda, declaro que la tésis titulada, `Detección de neo antígenos utilizando aprendizaje profundo en el marco del desarrollo de vacunas personalizadas en la inmunoterapia del Cáncer' y el trabajo presentado en este son de mi propiedad intelectual y confirmo que:

\begin{itemize} 
\item[\tiny{$\blacksquare$}] Este trabajo fue desarrollado durante mi candidatura a grado de doctor de esta universidad.
 
\item[\tiny{$\blacksquare$}] Ninguna parte de esta tésis ha sido presentado para otro grado de esta universidad o cualquier otra institución.
 
\item[\tiny{$\blacksquare$}] Cuando cito a otros autores, las fuentes has sido brindadas y con excepción de estas citas, mi trabajo es de mi autoría.
 
\item[\tiny{$\blacksquare$}] He agradecido las principales fuentes de ayuda.
 
\item[\tiny{$\blacksquare$}] En caso de que mi tesis haya sido desarrollado con un equipo de trabajo, yo he sido claro y he detallada la parte exacta de mi autoría.
\\
\end{itemize}
 
 
Firma:\\
\rule[1em]{25em}{0.5pt}  % This prints a line for the signature
 
Fecha:\\
\rule[1em]{25em}{0.5pt}  % This prints a line to write the date
}
\clearpage  % Declaration ended, now start a new page

%% ----------------------------------------------------------------
% The "Funny Quote Page"
\pagestyle{empty}  % No headers or footers for the following pages

\null\vfill
% Now comes the "Funny Quote", written in italics
\textit{``Con fe, disciplina y desinteresada devoción al deber, no hay nada que merezca la pena que no puedas lograr.''}

\begin{flushright}
Muhammad Ali Jinnah
\end{flushright}

\vfill\vfill\vfill\vfill\vfill\vfill\null
\clearpage  % Funny Quote page ended, start a new page
%% ----------------------------------------------------------------


\pagestyle{empty}  % Page style needs to be empty for this page
\dedicatory{Dedico este trabajo a mi esposa Pamela Laguna Laura, quien me ha acompañado durante todo este proceso, me ha motivado y sobre todo me ha dado su amor, que me ha ayudado a prevalecer y siempre seguir adelante. De igual forma, a mis padres Vicente Machaca Chino y Victoria Arceda Arenas, de ellos he aprendido el valor de la disciplina, la fuerza por emprender y la importancia de los valores sin importar las circunstancias; gracias a ellos he logrado cumplir mis objetivos. }

\clearpage 

% The Abstract Page
\addtotoc{Resumen}  % Add the "Abstract" page entry to the Contents
\abstract{
\addtocontents{toc}{\vspace{1em}}  % Add a gap in the Contents, for aesthetics

El cáncer es el mayor problema de salud mundial en la actualidad, frente a esto han surgido nuevos tratamientos basados en inmunoterapia como el desarrollo de vacunas personalizadas basadas en neoantígenos. Sin embargo, el proceso para identificar neoantígenos, es complejo y existen varias etapas para lograrlo desde el secuenciamiento de muestras tumorales, alineamiento con muestras de tejido saludable, identificación y anotación de mutaciones, para luego proseguir con la predicción de la unión de péptidos con el MHC y posteriormente la unión del pMHC con el TCR. Si esta unión procede, el péptido en cuestión es un fuerte candidato a neoantígeno. En este proceso, una de las fases más críticas es la predicción de la unión pMHC, lo cuál ha motivado el desarrollo de esta tesis. 

Además, las redes neuronales \textit{Transformers} han revolucionado el campo del procesamiento natural del lenguaje, y se han aplicado en muchas otras áreas como en Proteómica. Esto porque las proteínas al ser representadas como secuencias de aminoácidos, son muy similares a las secuencias de palabras en una oración. Es así, que otras investigaciones han aplicado el uso de Transformers y redes neuronales con mecanismos de atención para la predicción de la unión pMHC. Adicionalmente, existen modelos pre-entrenados como TAPE, ProtBert y ESM2, estos han sido entrenados con grandes volúmenes de datos para varias tareas de Proteómica. Basados en lo anterior, en esta tesis se propone el uso de aplicar fine-tuning a TAPE, ProtBert, ESM2(t6), ESM2(t12), ESM2(t30) y ESM2(t33) para la tarea de predicción de la unión pMHC, el fine-tunning consistió en agregar un bloque BiLSTM al final del modelo. Además, se ha evaluado el uso de \textit{Gradient Accumulation Steps} (GAS) y una metodología de congelamiento de capas.

Luego de los experimentos, los modelos con mejores resultados fueron TAPE-GAS, que resultó de aplicar GAS a TAPE y ESM2(t6)-Freeze, que resultó de aplicar la metodología de congelamiento a ESM2. Finalmente, se compararon estos modelos con los métodos de mejor resultado en el estado del arte, tales como: NetMHCpan4.1, MHCflurry2.0, ACME, Anthem y MixMHCpred2.2. Al finalizar los experimentos, TAPE-GAS y ESM2-Freeze superaron a los otros métodos en \textit{accuracy}, AUC, \textit{presicion, f1-score} y MCC. 






}

\clearpage  % Abstract ended, start a new page
%% ----------------------------------------------------------------

\setstretch{1.3}  % Reset the line-spacing to 1.3 for body text (if it has changed)

% The Acknowledgements page, for thanking everyone
%\acknowledgements{
%\addtocontents{toc}{\vspace{1em}}  % Add a gap in the Contents, for aesthetics

%Dedico este trabajo a mis padres Vicente Machaca Chino y Victoria Arceda Arenas, de ellos he aprendido el valor de la disciplina, la fuerza por emprender y la importancia de los valores; gracias a ellos he logrado cumplir mis objetivos. De igual forma, dedico este trabajo a mi esposa Pamela Laguna Laura, quien me ha acompañado durante todo este proceso, me ha motivado a seguir y sobre todo me ha dado su amor, que me ha ayudado a prevalecer y siempre seguir adelante.  \\ 

%}
%\clearpage  % End of the Acknowledgements
%% ----------------------------------------------------------------

\pagestyle{fancy}  %The page style headers have been "empty" all this time, now use the "fancy" headers as defined before to bring them back


%% ----------------------------------------------------------------
\lhead{\emph{Contents}}  % Set the left side page header to "Contents"
\tableofcontents  % Write out the Table of Contents

%% ----------------------------------------------------------------
\lhead{\emph{List of Figures}}  % Set the left side page header to "List if Figures"
\listoffigures  % Write out the List of Figures

%% ----------------------------------------------------------------
\lhead{\emph{List of Tables}}  % Set the left side page header to "List of Tables"
\renewcommand{\listtablename}{Índice de tablas}
\renewcommand{\tablename}{Tabla}
\listoftables  % Write out the List of Tables

%% ----------------------------------------------------------------
\setstretch{1.5}  % Set the line spacing to 1.5, this makes the following tables easier to read
\clearpage  % Start a new page
\lhead{\emph{Abreviaciones}}  % Set the left side page header to "Abbreviations"
\listofsymbols{ll}  % Include a list of Abbreviations (a table of two columns)
{
% \textbf{Acronym} & \textbf{W}hat (it) \textbf{S}tands \textbf{F}or \\
%\textbf{LAH} & \textbf{L}ist \textbf{A}bbreviations \textbf{H}ere \\

\textbf{ANN}		& Artificial Neural Network \\
\textbf{AUC}		& Area Under the Curve\\
\textbf{BERT}   & Bidirectional Encoder Representations from Transformers \\

\textbf{bp}		& Base pair in DNA \\
\textbf{CNN}		& Convolutional Neural Network \\
\textbf{DNN}		& Deep Neural Network \\
\textbf{DNA}		& Deoxyribonucleic Acid \\

\textbf{GNN}		&  Graph Neural Netowrk\\
\textbf{G-BERT}		&  Graph Bidirectional Encoder Representations from Transformers\\
\textbf{HLA}		& Human Leukocyte Antigens 		\\
\textbf{MCC} 		& Matthews Correlation Coefficient \\
\textbf{MHC-I}		& Major Histocompatibility Complex Class I		\\
\textbf{MHC-II}		& Major Histocompatibility Complex Class II		\\
\textbf{MHC-III}		& Major Histocompatibility Complex Class III		\\
\textbf{mRNA}		& Messenger Ribonucleic Acid \\
\textbf{NLP}		& Natural Language Processing	\\
\textbf{pMHC}		& Peptide-MHC ligand\\
\textbf{pMHC-TCR}    & pMHC T-cell receptor ligand\\
\textbf{RNA}		& Ribonucleic Acid \\
\textbf{RoBERTa}     & Optimized BERT \\
\textbf{RSL}     & Revisión Sistemática de la Literatura \\
\textbf{tRNA}		& Transfer Ribonucleic Acid \\
\textbf{TCR}			& T-cell receptor \\

}
\clearpage

%% ----------------------------------------------------------------
%\clearpage  % Start a new page
%\lhead{\emph{Physical Constants}}  % Set the left side page header to "Physical Constants"
%\listofconstants{lrcl}  % Include a list of Physical Constants (a four column table)
%{
% Constant Name & Symbol & = & Constant Value (with units) \\
%Speed of Light & $c$ & $=$ & $2.997\ 924\ 58\times10^{8}\ \mbox{ms}^{-\mbox{s}}$ (exact)\\

%}

%% ----------------------------------------------------------------
%\clearpage  %Start a new page
%\lhead{\emph{Symbols}}  % Set the left side page header to "Symbols"
%\listofnomenclature{lll}  % Include a list of Symbols (a three column table)
%{
% symbol & name & unit \\
%$a$ & distance & m \\
%$P$ & power & W (Js$^{-1}$) \\
%& & \\ % Gap to separate the Roman symbols from the Greek
%$\omega$ & angular frequency & rads$^{-1}$ \\
%}
%% ----------------------------------------------------------------
% End of the pre-able, contents and lists of things
% Begin the Dedication page

\setstretch{1.3}  % Return the line spacing back to 1.3

%\pagestyle{empty}  % Page style needs to be empty for this page
%\dedicatory{Dedico este trabajo a mis padres Vicente Machaca Chino y Victoria Arceda Arenas, de ellos he aprendido el valor de la disciplina, la fuerza por emprender y la importancia de los valores; gracias a ellos he logrado cumplir mis objetivos. De igual forma, dedico este trabajo a mi esposa Pamela Laguna Laura, quien me ha acompañado durante todo este proceso, me ha motivado a seguir y sobre todo me ha dado su amor, que me ha ayudado a prevalecer y siempre seguir adelante. \ldots}

\addtocontents{toc}{\vspace{2em}}  % Add a gap in the Contents, for aesthetics


%% ----------------------------------------------------------------
\mainmatter	  % Begin normal, numeric (1,2,3...) page numbering
\pagestyle{fancy}  % Return the page headers back to the "fancy" style

% Include the chapters of the thesis, as separate files
% Just uncomment the lines as you write the chapters

%% ------------------------------------------------------------------- %%
%% ------------------------------------------------------------------- %%
\chapter{Introducción}
\label{cap:introduccion}
\lhead{\emph{Introducción}}  % Change the left side page header to "Bibliography"


%% ------------------------------------------------------------------- %%
\section{Motivación}
\label{sec:motivacion}
El cáncer representa el mayor problema de salud mundial \citep{siegel2022cancer} y es el causante líder de muertes, solo en el 2020 se registraron alrededor de 10 millones de muertes y aproximadamente cada año 400000 niños desarrollan cáncer \citep{whocancer2022}. Lamentablemente, a pesar de muchos esfuerzos por mitigar las muertes causadas por esta enfermedad, los métodos tradicionales basados en cirugías, radioterapias y quimioterapias tienen baja efectividad \citep{peng2019neoantigen}. En este contexto, surge el desarrollo de la inmunoterapia del cáncer, el cuál tiene el objetivo estimular el sistema inmune de un paciente. La idea es que nuestro propio sistema inmune sea capaz de reconocer las células de cáncer como agentes extraños y por consiguiente elimine dichas células. Existen varios enfoques y metodologías en la inmunoterapia del cáncer, de estos, la de mayor estudio y efectividad es el desarrollo de vacunas personalizadas \citep{borden2022cancer}.

El desarrollo de vacunas personalizadas contra el cáncer es un proceso largo y depende de una correcta detección de neo antígenos. Estos neo antígenos son péptidos\footnote{Secuencias cortas de aminoacidos.} que solo se presentan en células cancerosas; entonces, el objetivo es entrenar a los linfocitos (células T) de un paciente para que estos puedan reconocer los neo antígenos y asi activar el sistema inmune.

Determinar qué estrategia o método de detección de neo antígenos es el adecuado o en qué circunstancias conviene la aplicación de alguno, es muy importante para el desarrollo de vacunas personalizadas \citep{de2020neoantigen, peng2019neoantigen}.  Sin embargo, a pesar de los esfuerzos de los investigadores en desarrollar métodos y herramientas,  menos del 3\% de los neo antígenos detectados logran activar a las células T (sistema inmune) \citep{de2020neoantigen}. De esta forma, es relevante que se continue con la investigación y desarrollo de nuevos métodos que permitan detectar neo antígenos.


\section{Problema}
\label{sec:problema}

Los neo antígenos son peṕtidos mutados específicos de tumores y son considerados los principales causantes de una respuesta inmune \citep{borden2022cancer, chen2021challenges, gopanenko2020main}. Es así que surgen varios esfuerzos e investigación en la Inmunoterapia del cáncer, concentradas en el estudio y detección de neo antígenos. En la actualidad existen tres clases de tratamientos basados en la representación y expresión de neo antígenos: vacunas personalizadas, terapias adoptivas de células T y \textit{immune checkpoint inhibitors}. De los métodos mencionados anteriormente, el desarrollo de vacunas personalizadas es considerado uno de los métodos con mayor probabilidad de éxito \citep{borden2022cancer}. Incluso varias compañías como BioNTech, Genocea Biosciences, Neon Therapeutics y Gritstone Oncology realizan investigación y ofrecen el servicio de generar vacunas personalizadas a pacientes de cáncer.

Según lo mencionado anteriormente, la detección de neo antígenos es un factor clave en el desarrollo de vacunas personalizadas. En este proceso el compuesto \textit{Major Histocompatibility Complex} (MHC), juega un papel muy importante, es el encargado de presentar los péptidos a la células T \citep{hashemi2022improved}. Para el caso de células humanas el gen MHC es conocido como Human Leukocyte Antigens (HLA) y es polimórfico, se cree que existen las 10000 diferentes \textit{HLA-I alleles} \citep{abelin2017mass}, esto complica mucho más la detección de neo antígenos. 

El ciclo de vida de un neo antígeno para células con núcleo podría resumirse como: primero una proteína es degradada en péptidos en el citoplasma de las células, luego los péptidos se enlazan a la molecula MHC (\textit{pMHC binding}), luego este compuesto sigue un trayecto hasta llegar a la membrana de la célula (\textit{pMHC presentation}), finalmente el compuesto pMHC es reconocido por  el T-cell Receptor (TCR) de las células T y así si activaría el sistema inmune. Además, el número de posibles péptidos enlazables a MHC  son entre 1000 a 10000, esto es el 0.1\% de los posibles péptidos  de 9 aminoacidos\footnote{La mayoría de péptidos enlazados a moléculas MHC-I tienen 9 aminoácidos, se suele utilizar el termino \textit{n-mer} para referirse a péptidos de \textit{n} aminoácidos.} \citep{abelin2017mass}. En este proceso, el objetivo es detectar los péptidos (neo antígenos) que llegan a la membrana de la célula, luego con ayuda de procedimientos de biotecnología, se entrena a las células T de un paciente para que aprenda a reconocer los neo antígenos.


El problema de \textit{pMHC binding} está casi solucionado con una precisión de 0.98 por parte de la herramienta NetMHCPan 4.1 \citep{reynisson2020netmhcpan}. Sin embargo, no es bueno limitar la detección de neo antígenos solo al problema de \textit{pMHC binding}, porque la mayoría de estos compuestos no llegan a la membrana \citep{mill2022neoms}, a este problema se le conoce como \textit{pMHC presentation}. Por ejemplo, se sabe que menos del 5\% de péptidos detectados llegan a la membrana \citep{de2020neoantigen, mill2022neoms, bulik2019deep, bassani2015mass, yadav2014predicting}. Además, existen herramientas como NeyMHC, NetMHCpan y MHCFlurry que tienen un buen desemepeño en \textit{pMHC binding}, pero con resultados pobres en  \textit{pMHC presentation} \citep{bulik2019deep}.



\subsection{Formulación del problema}

Menos del 5\% de péptidos detectados en \textit{pMHC binding}, llegan a la membrana de la células, para que luego sean reconocidos por las células T.  El proceso por el cúal un péptido enlazado a MHC llegue a la membrana es conocido como  \textit{pMHC presentation}, pero en este problema las propuestas recientes solo llegan a un 0.61 de precisión y 0.4 de \textit{recall}. En este contexto, la tesis se enfoca en el problema de \textit{pMHC presentation}, considerándolo como un problema de clasificación binaria, y tomando como entrada la secuencia de aminoácidos del péptido y la secuencia de aminoácidos de la proteína MHC. 



%% ------------------------------------------------------------------- %%
\section{Objetivos}
\label{sec:objetivos}

\subsection{Objetivo General}

Proponer un método basado en \textit{deep learning} para la detección de neo antígenos, enfocados en el problema de \textit{pMHC presentation}. 

\subsection{Objetivos específicos}

\begin{enumerate}[(a)]
\item Realizar una revisión sistemática de la literatura e implementar los métodos con mejor desempeño en la detección de neo antígenos.
\item Proponer e implementar un método basado en \textit{deep learning} para la detección de neo antígenos.		
\item Evaluar el método propuesto en bases de datos publicas.
\end{enumerate}

%% ------------------------------------------------------------------- %%
\section{Contribuciones}
\label{sec:contribuciones}
Las principales contribuciones de este trabajo son:

\begin{enumerate}[(a)]
	\item Una .........................;
	\item Una .........................;
	\item Una .........................;
	
\end{enumerate}

%% ------------------------------------------------------------------- %%
\section{Organización del Trabajo}
\label{sec:organizaciondeltrabajo}
En el Capítulo~\ref{cap:marcoteorico} se presentan los conceptos básicos ....

en el Capítulo~\ref{cap:estadodelarte} se describen los trabajos relacionados a la presente tesis.

..............................

Finalmente, en el Capítulo~\ref{cap:conclusiones} son expuestos las conclusiones del presente trabajo así como 
tambien las direcciones para continuar con el mismo en la sección de trabajos futuros.

%Em anexos está uma exemplificação da especificação da arquitetura, 
%neste caso é descrita a especificação da arquitetura 
%DAS-3 (Apêndice \ref{ape:especificacoes}).


%%%%%
 % Introduction
%% ------------------------------------------------------------------- %%
%% ------------------------------------------------------------------- %%
%% ------------------------------------------------------------------- %%
%% ------------------------------------------------------------------- %%
\chapter{Marco Conceptual}
\label{cap:marcoteorico}

\lhead{\emph{Marco Conceptual}} 

En este capítulo se presentarán conceptos necesarios para el correcto 
entendimiento de esta tesis.

%% ------------------------------------------------------------------- %%
%% ------------------------------------------------------------------- %%
%% ------------------------------------------------------------------- %%
%% ------------------------------------------------------------------- %%
%% ------------------------------------------------------------------- %%
%% ------------------------------------------------------------------- %%

%% -------------------------------------------------------------------- %%
%% -------------------------------------------------------------------- %%



%EOF
 % Introduction
%% ------------------------------------------------------------------- %%
\chapter{Estado del Arte}
\label{cap:estadodelarte}
\lhead{\emph{Estado del Arte}} 

En este capítulo presentaremos los resultados de la Revisión Sistemática de la Literatura (RSL) referente a los métodos de detección de neo antígenos con técnicas de \textit{deep learning} y desde una perspectiva en las ciencias de la computación.

%% ------------------------------------------------------------------- %%
%% ------------------------------------------------------------------- %%
%% ------------------------------------------------------------------- %%
%% ------------------------------------------------------------------- %%

\section{Revisión Sistemática de la Literatura (RSL)}

Con el objetivo de mapear las principales técnicas de detección de  neo antígenos, se planteó desarrollar una Revisión Sistemática de la Literatura (RSL). La RSL, se enfocó en los métodos basados en \textit{deep learning} y desde una perspectiva de las ciencias de la computación. Se definió este objetivo, porque en la literatura ya existían varios otros \textit{reviews}, enfocados en el proceso general de vacunas personalizadas, y detección de neo antígenos. En esta sección, se describe el proceso que se llevó a cabo y sus resultados.


\subsection{Cadenas de busqueda y bases de datos}

En la Tabla \ref{tab:key_words}, se presentan las cadenas de búsqueda utilizadas para la RSL. Generalmente los términos sinónimos a \textit{neoantigen} utilizados en la literatura son \textit{peptide} y \textit{epitope}. Luego, algunos trabajos se enfocan en predecir el enlace entre un péptido y la molécula MHC, pero para células humanas la molécula MHC tiene el nombre de HLA. Además, hay varias clases como MHC-I y MHC-II. Debido a eso, se tenía que considerar todos esos sinónimos de MHC. También, otra diferencia existe en el término ``enlace'', del enlace péptido con MHC, algunos trabajos se refieren a él con los términos: \textit{binding}, \textit{presentation}, \textit{prediction} y \textit{detection}. Finalmente, algunos trabajos se enfocan en otra fase de la detección de neo antígenos, esta consiste en predecir el enlace entre el compuesto pMHC y T-cell Receptor (TCR) de las células T.

Luego, se utilizó Google Schoolar y Mendeley como motores de búsqueda al ser estos unos motores que indexan casi la totalidad de artículos científicos. Utilizando estas herramientas, se obtuvo artículos de las bases de datos descritas en la Tabla \ref{tab:bd_RSL}.




\begin{table}[H]
	\begin{center}
		\caption{Cadenas de busqueda utilizadas en la RSL.}
		\label{tab:key_words}
		\setlength{\tabcolsep}{0.5em} % for the horizontal padding
		{\renewcommand{\arraystretch}{1.4}% for the vertical padding
				\begin{tabular}{p{14cm}}
				\textbf{Cadena de busqueda} \\ \hline
				neoantigen  AND (detection OR pipeline) AND deep learning                                                                               \\
				(MHC OR HLA) AND binding  AND deep learning                                                                                             \\				
				(MHC-I OR MHC-II OR MHC OR HLA) AND (peptide OR epitope) AND ( binding OR affinity OR prediction OR detection OR presentation)          \\
				TCR interaction prediction                                                                                                              \\		
			\end{tabular}
		}
	\end{center}
\end{table}

\begin{table}[H]
	\begin{center}
		\caption{Bases de datos utilizadas en la RSL.}
		\label{tab:bd_RSL}
		\setlength{\tabcolsep}{0.5em} % for the horizontal padding
		{\renewcommand{\arraystretch}{1.2}% for the vertical padding
			\begin{tabular}{p{14cm}}
				\textbf{Bases de datos} \\ \hline
				IEEE Xplore                                                                               \\
				Science Direct \\				
				Springer          \\
				ACM Digital Library                                                                                                             \\	
				PubMed \\ 
				BioRxiv \\ 	
			\end{tabular}
		}
	\end{center}
\end{table}

\subsection{Selección de artículos}

Con las cadenas de búsqueda y considerando solo los artículos desde el 2018, se analizó el título de cada artículo encontrado por los motores de búsqueda y se seleccionaron 334 artículos. En la Tabla \ref{tab:number_papers}, se presenta la cantidad de artículos publicados por año. Para el caso del 2022, solo se tienen 57 artículos porque esta tesis se redactó a mediados del año 2022. 

Del total de artículos encontrados (342), se seleccionó un subconjunto basado en los criterios de inclusión y exclusión presentados de la Tabla  \ref{tab:criterios}. Estos criterios incluían que el artículo pertenezca a un \textit{conference} o \textit{journal} reconocido, que tenga una metodología detallada y que pertenezca al area de ciencia de la computación. Luego, en la Tabla \ref{tab:criterios}, se puede ver que hay un puntaje según cada criterio de inclusión, se utilizó este puntaje para calificar cada artículo y luego se seleccionaron los artículos que tenían un puntaje mayor a 4. En este proceso, se analizó el \textit{abstract} de los artículos y ciertas partes importantes según era necesario para asignar el puntaje. Al finalizar esta etapa, se obtuvieron 253 artículos, estos son los trabajos que se han analizado en la RSL. Adicionalmente, a los artículos seleccionados, se han considerado otros trabajos importantes que proponian bases de datos, \textit{pipelines} y \textit{reviews}.




\begin{table}[H]
	\begin{center}
		\caption{Cantidad de artículos encontrados y seleccionados según los criterios de inclusión y exclusión en la RSL.}
		\label{tab:number_papers}
		\setlength{\tabcolsep}{0.5em} % for the horizontal padding
		{\renewcommand{\arraystretch}{1.2}% for the vertical padding
			\begin{tabular}{ccc}
					\textbf{Año} & \textbf{Artículos encontrados} & \textbf{Artículos seleccionados}\\ \hline
				    2018 & 53 & 42 \\
				    2019 & 79 & 52 \\
				    2020 & 81 & 67 \\
				    2021 & 64 & 51 \\
				    2022 & 57 & 41 \\ \hline
				    Total & \textbf{342} & \textbf{253} \\
			\end{tabular}
		}
	\end{center}
\end{table}

\begin{table}[H]
	\begin{center}
		\caption{Criterios de inclusión y exclusión de artículos utilizados en la RSL.}
		\label{tab:criterios}
		\setlength{\tabcolsep}{0.5em} % for the horizontal padding
		{\renewcommand{\arraystretch}{1.2}% for the vertical padding
			\begin{tabular}{p{5.5cm}p{5.5cm}c}
				\textbf{Criterios de inclusión}                                                   & \textbf{Criterios de exclusión}                                                          & \textbf{Puntaje} \\ \hline
				Artículos con categoría ERA (A, B o C) si son conferencias y Journals Q1, Q2 o Q3. & No considerar los trabajos de baja calidad, que no esten rankeados.                       & 3                \\
				Trabajos que se basen en \textit{deep learning} para la detección de neo antígenos.          & Trabajos que se basan en el uso de alguna herramienta (investigaciónes realizadas por cientificos de otras areas). & 2                \\
				La metodología es detallada.                                                       &                                                                                          & 2                \\
				Tiene resultados clínicos                                                         &                                                                                          & 2                \\
				Tiene repositorio de código fuente.                                          &                                                                                          & 1                \\
				Comparte la base de datos utilizada.                                         &                                                                                          & 1               
			\end{tabular}
		}
	\end{center}
\end{table}





\section{Resultados de la RSL}\index{} 
\label{sec:neoantigen}


 El proceso para la detección de neo antígenos, es complejo, y generalmente consiste en: (1) extracción del tejido tumoral y secuenciamiento, (2) identificación de mutaciones, (3) detección de péptidos como resultado de alineamiento con muestras sanas, (4) predicción de \textit{peptide-MHC binding (pMHC)}, (5) predicción de \textit{pMHC presentation} y (6) predicción del enlace pMHC-TCR \citep{de2020neoantigen, peng2019neoantigen}. De este proceso, la mayoría de investigaciones se centra en el problema de \textit{peptide-MHC binding}, \textit{peptide-MHC presentation} y predicción del enlace pMHC-TCR. Entonces, se va a reportar los trabajos relacionados según esta clasificación. Tambien, se van a incluir en otra clasificación, los pipelines que integran varias herramientas para todo el proceso de detección de neo antígenos; Investigaciones que presentan bases de datos; y finalmente \textit{reviews} relacionados a la tesis.


\subsection{\textit{Reviews}}

La detección de neo antígenos es un problema interdisciplinar y esto ha originado  varios \textit{reviews} desde diferentes perspectivas. Entonces se ha planteado la siguiente clasificación: basados en \textit{Next-Generation Sequencing}, \textit{Mass Spectrometry}, interacción \textit{peptide-MHC}, basados en información estructural, enfocados en TCR, buenas prácticas y los enfocados en el proceso completo de generación de vacunas personalizadas.

Primero, presentamos los trabajos que se enfocan en estudios de \textit{Next-Generation Sequencing} (Tabla \ref{tab:review_seq}), para la detección de neo antígenos e inmunoterapia del Cáncer. Estos trabajos principalmente utilizan información secuencial de \textit{DNA} y gracias a las tecnologías modernas ahora se pueden considerar las secuencias de \textit{RNASeq}. Las tecnologías de RNASeq, proveen información mas precisa de la transcripción e identificación de isoformos que otros métodos \citep{wang2009rna}. Mayormente, estas tecnologías se limitan a algoritmos alineamiento con genomas de referencia \citep{groisberg2018immunotherapy}. 




%%%%%%%%%%%%%%%%%%%%%%%%%%%%%%%%%%%%%%%%%%%%%%%%%%%%%%%%%%%%%%%%%%%%%%%%%
%PENDIENTE queda pendinete agregar ms información
%%%%%%%%%%%%%%%%%%%%%%%%%%%%%%%%%%%%%%%%%%%%%%%%%%%%%%%%%%%%%%%%%%%%%%%%%

\begin{table}[H]
		\caption{Listado de los \textit{reviews}, que se enfocan en estudios de \textit{Next-Generation Sequencing} para la detección de neo antígenoes e inmunoterapia del Cáncer.}
	\label{tab:review_seq}
	\begin{tabular}{p{3cm}p{10cm}}
	\textbf{Autor-año }                            & \textbf{Título}                                                                                                                                \\ \hline
		\cite{zhou2022comprehensive}     & A Comprehensive Survey of Genomic Mutations in Breast Cancer Reveals Recurrent Neoantigens as Potential Therapeutic Targets            \\
		\cite{battaglia2020neoantigen}   & Neoantigen prediction from genomic and transcriptomic data                                                                             \\
		\cite{mirandola2020quest}        & The Quest for the Next-Generation of Tumor Targets: Discovery and Prioritization in the Genomics Era                                   \\		
		\cite{groisberg2018immunotherapy}& Immunotherapy and next-generation sequencing guided therapy for precision oncology: what have we learnt and what does the future hold?
	\end{tabular}
\end{table}

Algunos trabajos son más específicos, y se enfocan en la interacción de un péptido y la molécula MHC. Esta interacción es un factor clave, porque si se forma el enlace pMHC y luego este compuesto es presentado a las células T, es posible activar el sistema inmune. En la Tabla \ref{tab:review_mhc}, se presenta estos \textit{reviews}. La mayoría de estos trabajos, se centran en la molécula MHC-I \citep{mateo2020comparison, mei2020comprehensive, schmidt2019mhc, mei2020comprehensive} , molécula MHC-II \citep{jensen2018improved} y todos los tipos de MHC en general \citep{nielsen2020immunoinformatics, liu2020review, liu2020review}. También, hay trabajos que estudian la complejidad de esta molécula y todos sus \textit{alleles} \citep{radwan2020advances}.



\begin{table}[H]
	\caption{Listado de los \textit{reviews}, que se enfocan en estudios de la interacción de péptidos y la molécula MHC, para la detección de neo antígenoes.}
	\label{tab:review_mhc}
	\begin{tabular}{p{3cm}p{10cm}}
		\textbf{Autor-año }                            & \textbf{Título}                                                                                                                                \\ \hline
		\cite{mateo2020comparison}          & Comparison of machine learning models for the prediction of cancer cells using MHC class I complexes                 \\
		\cite{mei2020comprehensive}         & A comprehensive review and performance evaluation of bioinformatics tools for HLA class I peptide-binding prediction \\
		
		\cite{nielsen2020immunoinformatics} & Immunoinformatics: predicting peptide–MHC binding                                                                    \\
		\cite{liu2020review}                & A review on the methods of peptide-MHC binding prediction                                                            \\
		\cite{paul2020major}                & Major histocompatibility complex binding, eluted ligands, and immunogenicity: benchmark testing and predictions      \\
	
		\cite{radwan2020advances}           & Advances in the Evolutionary Understanding of MHC Polymorphism 
		
		                                                      \\
		
		\cite{schmidt2019mhc}               & MHC class I presented antigens from malignancies: A perspective on analytical characterization \& immunogenicity     \\
	
		\cite{jensen2018improved}           & Improved methods for predicting peptide binding affinity to MHC class II molecules                                   \\
		
		\cite{mei2020comprehensive}         & A comprehensive review and performance evaluation of bioinformatics tools for HLA class I peptide-binding prediction \\
		\cite{liu2020review}                & A review on the methods of peptide-MHC binding prediction                                                            \\
		\cite{paul2020major}                & Major histocompatibility complex binding, eluted ligands, and immunogenicity: benchmark testing and predictions      \\
		\cite{radwan2020advances}           & Advances in the Evolutionary Understanding of MHC Polymorphism                                                       \\
		\cite{schmidt2019mhc}               & MHC class I presented antigens from malignancies: A perspective on analytical characterization \& immunogenicity     \\
		\cite{jensen2018improved}           & Improved methods for predicting peptide binding affinity to MHC class II molecules                                  
	\end{tabular}
\end{table}

La mayoría de \textit{reviews} estudian las técnicas basadas en secuencias de DNA y RNA, pero recientemente se está utilizando \textit{Mass spectrometry}, para secuenciar los péptidos y moléculas MHC ya enlazados y presentes en las membranas de las células. Este avance ha impulsado la creación de nuevas bases de datos y métodos para el problema de \textit{peptide-MHC presentation}. En este contexto, en la Tabla \ref{tab:review_ms}, se presentan todos los \textit{reviews}, enfocados en estudiar \textit{Mass spectrometry} para la detección de neo antígenos.



\begin{table}[H]
	\caption{Listado de los \textit{reviews}, que se enfocan en estudios de \textit{Mass spectrometry} para la detección de neo antígenoes.}
	\label{tab:review_ms}
	\begin{tabular}{p{3cm}p{10cm}}
	\textbf{Autor-año }                            & \textbf{Título}                                                                                                                                 \\ \hline
		\cite{kote2020mass}           & Mass spectrometry-based identification of MHC-associated peptides                                                          \\
		\cite{kote2020mass}           & Mass spectrometry-based identification of MHC-associated peptides                                                          \\
		\cite{zhang2019application}   & Application of mass spectrometry-based MHC immunopeptidome profiling in neoantigen identification for tumor immunotherapyA \\
		\cite{chen2021identification} & Identification of MHC peptides using mass spectrometry for neoantigen discovery and cancer vaccine development             \\
		\cite{creech2018role}         & The role of mass spectrometry and proteogenomics in the advancement of HLA epitope prediction                              \\
		\cite{zhang2019application}   & Application of mass spectrometry-based MHC immunopeptidome profiling in neoantigen identification for tumor immunotherapyA \\
		\cite{creech2018role}         & The role of mass spectrometry and proteogenomics in the advancement of HLA epitope prediction                             
	\end{tabular}
\end{table}


En si la detección de neo antígenos, es un proceso muy largo e integra métodos de secuenciamiento, alineamiento, detección de mutaciones, identificación de péptidos, predicción de la interacción \textit{peptide-MHC}, y finalmente el trabajo biotecnológico para la generación de vacunas. Entonces, en la Tabla \ref{tab:review_2022_2021} y \ref{tab:review_2020_2019}, se presenta el lista de \textit{reviews}, que explican el problema de generación de vacunas pero desde una vista panoramica incluyendo todo el proceso completo. Algunos trabajos se enfocan en demostrar la posibilidad de crear vacunas personalizadas contra en Cáncer \citep{lang2022identification, richard2022neoantigen, pao2022therapeutic, reynolds2022neoantigen, mccaffrey2022bioinformatic, fritsch2020personal} y otros trabajos, priorizan la importancia de los neo antígenos \citep{okada2022identification, zheng2022neoantigen, wang2021gene, pearlman2021targeting, arnaud2020biotechnologies, han2020progress}.


\begin{table}[H]
	\caption{Listado de los \textit{reviews}, que se enfocan en presentar en proceso general de detección de neo antígenoes y vacunas personalizadas del año 2022 y 2021.}
	\label{tab:review_2022_2021}
	\begin{tabular}{p{3cm}p{10cm}}
		\textbf{Autor-año }                            & \textbf{Título}                                                                                                                             \\ \hline
		\cite{tran2022tale}                   & A tale of solving two computational challenges in protein science: neoantigen prediction and protein structure prediction                 \\
		\cite{lang2022identification}         & Identification of neoantigens for individualized therapeutic cancer vaccines                                                              \\
		\cite{okada2022identification}        & Identification of Neoantigens in Cancer Cells as Targets for Immunotherapy                                                                \\
		\cite{bollineni2022chasing}           & Chasing neoantigens; invite naïve T cells to the party                                                                                    \\
		\cite{richard2022neoantigen}          & Neoantigen-based personalized cancer vaccines: the emergence of precision cancer immunotherapy                                            \\
		\cite{pao2022therapeutic}             & Therapeutic Vaccines Targeting Neoantigens to Induce T-Cell Immunity against Cancers                                                      \\
		\cite{fang2022neoantigens}            & Neoantigens and their potential applications in tumor immunotherapy                                                                       \\
		\cite{zheng2022neoantigen}            & Neoantigen: A Promising Target for the Immunotherapy of Colorectal Cancer                                                                 \\
		\cite{redwood2022s}                   & What's next in cancer immunotherapy?-The promise and challenges of neoantigen vaccination                                                 \\
		\cite{reynolds2022neoantigen}         & Neoantigen Cancer Vaccines: Generation, Optimization, and Therapeutic Targeting Strategies                                                \\
		\cite{roesler2022beyond}              & Beyond Sequencing: Prioritizing and Delivering Neoantigens for Cancer Vaccines                                                            \\
		\cite{mccaffrey2022bioinformatic}     & Bioinformatic Techniques for Vaccine Development: Epitope Prediction and Structural Vaccinology                                           \\
		\cite{fotakis2021computational}       & Computational cancer neoantigen prediction: current status and recent advances                                                            \\
		\cite{wang2021beyond}                 & Beyond tumor mutation burden: tumor neoantigen burden as a biomarker for immunotherapy and other types of therapy                         \\
		\cite{ferreira2021glycoproteogenomics}& Glycoproteogenomics: Setting the Course for Next-generation Cancer Neoantigen Discovery for Cancer Vaccines                               \\
		\cite{blass2021advances}              & Advances in the development of personalized neoantigen-based therapeutic cancer vaccines                                                  \\
		\cite{wang2021gene}                   & Gene fusion neoantigens: Emerging targets for cancer immunotherapy                                                                        \\
		\cite{pearlman2021targeting}          & Targeting public neoantigens for cancer immunotherapy                                                                                     \\	                                                                    
	\end{tabular}
\end{table}

\begin{table}[H]
	\caption{Listado de los \textit{reviews}, que se enfocan en presentar en proceso general de detección de neo antígenoes y vacunas personalizadas del año 2020 y 2019.}
	\label{tab:review_2020_2019}
	\begin{tabular}{p{3cm}p{10cm}}
		\textbf{Autor-año }                            & \textbf{Título}                                                                                                                             \\ \hline	
		\cite{arnaud2020biotechnologies}      & Biotechnologies to tackle the challenge of neoantigen identification                                                                      \\
		\cite{fritsch2020personal}            & Personal neoantigen cancer vaccines: a road not fully paved                                                                               \\
		\cite{holtstrater2020bioinformatics}  & Bioinformatics for cancer immunotherapy                                                                                                   \\
		\cite{roudko2020computational}        & Computational prediction and validation of tumor-associated neoantigens                                                                   \\
		\cite{esprit2020neo}                  & Neo-antigen mRNA vaccines                                                                                                                 \\
		\cite{chen2020personalized}           & Personalized neoantigen vaccination with synthetic long peptides: recent advances and future perspectives                                 \\
		\cite{londhe2020personalized}         & Personalized neoantigen vaccines: A glimmer of hope for glioblastoma                                                                      \\
		\cite{han2020progress}                & Progress in neoantigen targeted cancer immunotherapies                                                                                    \\
		\cite{keshavarzi2020ai}               & AI and Immunoinformatics                                                                                                                  \\
		\cite{jiang2019tumor}                 & Tumor neoantigens: from basic research to clinical applications                                                                           \\
		\cite{mardis2019neoantigens}          & Neoantigens and genome instability: impact on immunogenomic phenotypes and immunotherapy response                                         \\
		\cite{de2019advancing}                & Advancing cancer immunotherapy: a vision for the field                                                                                    \\
		\cite{li2018recent}                   & Recent updates in cancer immunotherapy: a comprehensive review and perspective of the 2018 China Cancer Immunotherapy Workshop in Beijing \\
		\cite{sidhom2018applications}         & Applications of Artificial Intelligence \& Machine Learning in Cancer Immunology                                                          \\
		\cite{doytchinova2018silico}          & In silico prediction of cancer immunogens: current state of the art                                                                      
	\end{tabular}
\end{table}


Gracias a \textit{Next-generation Secuencing} y \textit{Mass spectrometry}, se ha logrado muchon avances en la Bioinformática, pero a veces es necesario tener información adicional como las propiedades estructurales de los aminoacidos. Debido a esto, han surgido varias investigaciones y los \textit{reviews} de la Tabla \ref{tab:review_structure}, que explican como se pueden utilizar este tipo de propiedades para predecir la interacción pMHC. Lamentablemente, solo se ha identificado dos trabajos \citep{perez2022structural,antunes2018structure}, porque no se cuenta con muchas muestras de este problema.



\begin{table}[H]
	\caption{Listado de los \textit{reviews}, que se enfocan en estudios que utilizan propiedades estructurales de los aminoacidos para la de detección de neo antígenoes.}
	\label{tab:review_structure}
	\begin{tabular}{p{3cm}p{10cm}}
		\textbf{Autor-año }                            & \textbf{Título}                                                                                                                             \\ \hline	
		\cite{perez2022structural}  & Structural Prediction of Peptide–MHC Binding Modes                                                 \\
		\cite{antunes2018structure} & Structure-based methods for binding mode and binding affinity prediction for peptide-MHC complexes \\		
	\end{tabular}
\end{table}


Generalmente, con la predicción del enlace pMHC, podría terminar el trabajo Bioinformático, para luego proceder a los trabajos \textit{in vitro} e \textit{in vivo}. Pero, algunos trabajos, tambien buscan entender que hace que un cmpuesto pMHC se enlace al TCR y así se genere una respuesta inmune. Esto tambien ha generado bastantes \textit{reviews} presentados en la Tabla \ref{tab:review_tcr}.

\begin{table}[H]
	\caption{Listado de los \textit{reviews}, que se enfocan en estudios de la interacción de compuestos pMHC con TCR.}
	\label{tab:review_tcr}
	\begin{tabular}{p{3cm}p{10cm}}
		\textbf{Autor-año }                            & \textbf{Título}                                                                                                                             \\ \hline
		\cite{kast2021advances}     & Advances in identification and selection of personalized neoantigen/T-cell pairs for autologous adoptive T cell therapies           \\
		\cite{schaap2021t}          & T Cell Epitope Prediction and Its Application to Immunotherapy                                                                      \\                                    
		\cite{zvyagin2020overview}  & An overview of immunoinformatics approaches and databases linking T cell receptor repertoires to their antigen specificity          \\
		\cite{sidney2020epitope}    & Epitope prediction and identification- adaptive T cell responses in humans                                                          \\
		
		\cite{zvyagin2020overview}  & An overview of immunoinformatics approaches and databases linking T cell receptor repertoires to their antigen specificity          \\
		\cite{spear2019understanding} & Understanding TCR affinity, antigen specificity, and cross-reactivity to improve TCR gene-modified T cells for cancer immunotherapy \\		                                               
		\end{tabular}
	\end{table}


Finalmente, se han desarrollado \textit{reviews} que detallan los principales desafíos, buenas prácticas y perspectivas futuras en la detección de neo antígenos (Tabla \ref{tab:review_buen}). De estos trabajos, el \textit{review} de \cite{gopanenko2020main} y \cite{borden2022cancer}, explican detalladamente, todos los métodos de cada fase para la detección de neo antígenos; adicionalmente, explican las ventajas de cada método y los problemas actuales. También, resaltamos el trabajo de \cite{richters2019best}, que resalta las buenas prácticas de este campo de estudio.




\begin{table}[H]
	\caption{Listado de los \textit{reviews}, que se enfocan en presentar buenas prácticas en el proceso de detección de neo antígenoes y generación de vacunas personalizadas,}
	\label{tab:review_buen}
	\begin{tabular}{p{3cm}p{10cm}}
		\textbf{Autor-año }                            & \textbf{Título}                                                                                                                                \\ \hline
		\cite{borden2022cancer}       & Cancer Neoantigens: Challenges and Future Directions for Prediction, Prioritization, and Validation                                           \\
		\cite{chen2021challenges}     & Challenges targeting cancer neoantigens in 2021: a systematic literature review                                                             \\	
		\cite{gopanenko2020main}      & Main strategies for the identification of neoantigens                                                                                         \\
		\cite{de2020neoantigen}       & Neoantigen prediction and computational perspectives towards clinical benefit: recommendations from the ESMO Precision Medicine Working Group \\	
		
		\cite{richters2019best}       & Best practices for bioinformatic characterization of neoantigens for clinical utility                                                         \\
		\cite{garcia2019determinants} & Determinants for Neoantigen Identification                                                                                                    \\
		\cite{aurisicchio2018perfect} & The perfect personalized cancer therapy: cancer vaccines against neoantigens                                                                  \\
		\cite{barros2018immunological}& Immunological-based approaches for cancer therapy                                                                                             \\
		\cite{tureci2018challenges}   & Challenges towards the realization of individualized cancer vaccines                                                                          \\
		\cite{villani2018systems}     & Systems immunology: Learning the rules of the immune system                                                                                   \\
		\cite{richters2019best}       & Best practices for bioinformatic characterization of neoantigens for clinical utility                                                         \\
		\cite{garcia2019determinants} & Determinants for Neoantigen Identification                                                                                                    	                                                               \\
		\cite{barros2018immunological}& Immunological-based approaches for cancer therapy                                                                                            \\	
	                                                                                
	\end{tabular}
\end{table}



\subsection{\textit{Pipelines}}

Debido a la complejidad del proceso y la gran cantidad de métodos desarrollados, se ha desarrollado software y \textit{pipelines} que pretenden facilitar el uso de estas herramientas. Entre los \textit{pipelines} más conocidas antes del 2018 tenemos: Somaticseq \citep{fang2015ensemble}, CloudNeo \citep{bais2017cloudneo}, MuPeXI \citep{bjerregaard2017mupexi}, NeoepitopePred \citep{tran2015immunogenicity}, y NeoFuse \citep{gros2016prospective}. Estas herramientas en su mayoría toman como entrada archivos Variant Calling Files (VCF) y archivos de alineamiento BAM, para la detección de mutaciones (inserciones, eliminaciones y fusión de genes) y posibles neo antígenos. Luego, también hemos detallado, un conjunto de herramientas a partir del 2018, en la Tabla \ref{tab:review_pipelines}.




\begin{table}[H]
	\caption{Listado de \textit{pipelines} desde el 2018, para la detección de neo antígenos.}
	\label{tab:review_pipelines}
	\begin{tabular}{lp{2.5cm}p{4cm}p{4cm}}
		\textbf{Nombre} & \textbf{Autor-año}                                  & \textbf{Entrada}                                         & \textbf{Salida}                                     \\ \hline
		Neopepsee       & \cite{kim2018neopepsee}           & RNA-seq, somatic mutations (VCF), tipo de HLA (opcional) & Neo antígenos y niveles de expresión de los genes   \\
		PGV Pipeline    & \cite{rubinsteyn2018computational}& DNA-seq                                                  & Neo antígenos                                       \\
		ScanNeo         & \cite{wang2019scanneo}            & RNA-seq                                                  & Neo antígenos                                       \\
		NeoPredPipe     & \cite{schenck2019neopredpipe}     & Mutaciones (VCF) y tipo de HLA                           & Neo antígenos y anotación de variantes              \\
		pVACtools       & \cite{hundal2020pvactools}        & Mutaciones (VCF)                                         & Neo antígenos                                       \\
		ProGeo-neo      & \cite{li2020progeo}               & RNA-seq y somatic mutations (VCF)                        & Neo antígenos                                       \\
		neoepiscope     & \cite{wood2020neoepiscope}        & Somatic mutations (VCF) y archivos BAM                   & Neo antígenos y mutaciones                          \\
		neoANT-HILL     & \cite{coelho2020neoant}           & RNA-seq y somatic mutations (VCF)                        & Neo antígenos,  y niveles de expresión de los genes \\
		NAP-CNB         & \cite{wert2021predicting}         & RNA-seq                                                  & Neo antígenos                                       \\
		Valid-NEO       & \cite{terai2022valid}             & Somatic mutations (VCF), tipo de HLA (opcional)          & Neo antígenos                                      
	\end{tabular}
\end{table}


\subsection{\textit{Bases de datos}}

En la Tabla \ref{tab:bd}, presentamos una lista de bases de datos públicas. Estas bases de datos se centran en la interacción \textit{peptide-MHC} \citep{wu2018tsnadb, zhou2019neopeptide, tan2020dbpepneo, lu2022dbpepneo2} y pMHC con TCR \citep{shugay2018vdjdb, bagaev2020vdjdb}. También, hay un trabajo que presenta las estructuras 3D de las péptidos y HLA abriendo una nueva rama de investigación desde otro enfoque. Finalmente, la base de datos por excelencia IEDB \citep{vita2019immune}.



\begin{table}[H]
	\caption{Bases de datos públicas de \textit{pMHC binding}, \textit{pMHC presentation}, interacción pMHC-TCR y estructuras 3D de proteínas.}
	\label{tab:bd}
	\begin{tabular}{lp{3cm}p{8cm}}
		\textbf{Nombre} & \textbf{Autor-año}                                                                & \textbf{Descripción}                                                                                                                                                                                      \\ \hline
		VDJdb           & \cite{shugay2018vdjdb} y \cite{bagaev2020vdjdb}& Base de datos del enlace TCR con pMHC, cuenta con 5491 muestras                                                                                                                                           \\
		IEDB            & \cite{vita2019immune}                                           & La base de datos mas grande, contiene información \textit{T-cell epitopes} de humanos y otros organismos.                                                                                                          \\
		TSNAdb          & \cite{wu2018tsnadb}                                             & Contiene 7748 muestras de mutaciones y HLA de 16 tipos de Cáncer.                                                                                                                                         \\
		NeoPeptide      & \cite{zhou2019neopeptide}                                       & Contiene muestras de neo antígenos, resultado de mutaciones somáticas y artículos relacionados. Contiene 1818137 epitopes de ms de 36000 neo antígenos.                                                   \\
		pHLA3D          & \cite{e2019phla3d}                                              & Presenta 106 estructuras 3D de las cadenas alpha, $\beta_2M$ y peptidos de las moléculas HLA-I                                                                                                               \\
		dbPepNeo        & \cite{tan2020dbpepneo}                                          & Tiene muestras validadas del enlace \textit{peptide-MHC}, a partir de MS. Contiene 407794 muestras de baja calidad, 247 de mediana calidad y 295 muestras de alta calidad.                                         \\
		dbPepNeo2. 0    & \cite{lu2022dbpepneo2}                                          & Recolecta una lista de neo antígenos y moléculas HLA. Presenta 801 muestras de alta calidad y 842289  de mala calidad de HLAs. Tambien, 55 neo antígenos de clase II y 630 neo antígenos enlazados a TCR. \\
		IntroSpect      & \cite{zhang2022introspect}                                      & Herramienta para la construcción de bases de datos sobre \textit{peptide-MHC binding} . Utiliza datos de \textit{Mass Spectrometry}                                                                                         
	\end{tabular}
\end{table}


\subsection{\textit{Peptide-MHC binding}}

Existen herramientas de Software que se basan en la predicción del enlace entre las moléculas Major Histocompatibility Complex (MHC) y péptidos (posibles neo antígenos). La predicción de estos enlaces es importante para determinar qué péptidos pueden representar neo antígenos. Entre las principales propuestas que utilizan Regresión lineal y Redes Neuronales, tenemos: NetMHC4 \citep{stevanovic2017landscape}, NetMHCpan4 \citep{robbins2013mining}, PickPocket \citep{tran2014cancer}, NetMHCcons \citep{castle2012exploiting}, NetMHCIIpan \citep{yadav2014predicting}. También, existen alternativas como NeonMHC \citep{van2013tumor} que utilizan Redes Neuronales Convolucionales. Luego, otras propuestas se basan en la mejorar la predicción de un posible neo antígeno \citep{lu2021deep, hao2021improvement, lang2021neofox, chen2021identification, yang2021deepnetbim, li2021deepimmuno}.  Una desventaja de estos métodos, es referente a la necesidad de contar de antemano con posibles peptidos, esto complica una propuesta \textit{end-to-end} que tome como entrada una secuencia de ADN.\\



A pesar de la gran cantidad de métodos y herramientas no existe un método que pueda ser definido como el de mejor desempeño \citep{de2020neoantigen}, incluso a pesar de ya haberse desarrollado algunos \textit{benchmarkings}. Por ejemplo, en el 2015 se desarrolló una comparativa de los métodos SMM, ANN, ARB y NetMHCpan \citep{trolle2015automated}, sin ninguna conclusión sobresaliente. Luego en el 2018 y 2019 se vuelve a intentar realizar otra comparativa \citep{bonsack2019performance, zhao2018systematically}, sin lograr determinar a un método con mayor desempeño. Tambien se han desarrollado \textit{surveys} sobre como los métodos computacionales pueden tener beneficios clínicos \citep{de2020neoantigen} y sus principales desafios \citep{chen2021challenges}. \\

Finalmente, en la Tabla \ref{tab:review}, se presenta un resumen de los métodos basados en \textit{MHC-binding} y \textit{pipelines}. Tambien, indicamos cuales son \textit{open source}.\\

\begin{table}[H]
	\centering
	\caption{Resumen de los métodos de detección de neo antígenos.}
	\label{tab:review}
	\begin{tabular}{llll}
		\hline
		\textbf{Nombre} & \textbf{MHC-binding} & \textbf{Método} & \textbf{Open source} \\ \hline
		NetMHC4         & \checkmark            & ANN             &                      \\
		NetMHCpan4      & \checkmark            & ANN             &                      \\
		PickPocket      & \checkmark            & ANN             &                      \\
		NetMHCcons      & \checkmark            & ANN             &                      \\
		NetMHCIIpan     & \checkmark            & ANN             &                      \\
		NeonMHC         & \checkmark            & CNN             &                      \\
		DeepNetBim      & \checkmark            & Deep learning   & \checkmark            \\
		DeepImmuno      & \checkmark            & CNN             &                      \\
		NeoPredPipe     &                      & pipeline        & \checkmark            \\
		CloudNeo        &                      & pipeline        &                      \\
		MuPeXI          &                      & pipeline        &                      \\
		NeoepitopePred  &                      & pipeline        &                      \\
		Neoepiscope     &                      & pipeline        &                      \\
		pVACtools       &                      & pipeline        & \checkmark            \\
		NeoFuse         &                      & pipeline        & \checkmark    \\   \hline    
	\end{tabular}
\end{table}

%\subsection{\textit{Peptide-MHC presentation}}

%\subsection{Enlace pMHC-TCR}



 % Introduction
%% ------------------------------------------------------------------- %%
%% ------------------------------------------------------------------- %%
%% ------------------------------------------------------------------- %%
%% ------------------------------------------------------------------- %%
\chapter{Propuesta}
\label{cap:propuesta}

\lhead{\emph{Propuesta}} 
%% ------------------------------------------------------------------- %%
%% ------------------------------------------------------------------- %%
%% ------------------------------------------------------------------- %%
%% ------------------------------------------------------------------- %%
%% ------------------------------------------------------------------- %%
%% ------------------------------------------------------------------- %%

%% -------------------------------------------------------------------- %%
%% -------------------------------------------------------------------- %%


La detección de neoantígenos es un proceso largo, descrito anteriormente. Debido a esto, esta investigación se ha centrado en la predicción de la unión pMHC, porque es una de las etapas con mayor investigación en el estado del arte y sin embargo, los resultados aún carecen de buen desempeño. En resumen. En resumen en este trabajo hemos realizado fine-tuning a modelos Transformer pre-entrenados, para la tarea de predicción de la unión pMHC.




\begin{comment}
	

\section{Predicción de la afinidad peptido-MHC (peptide-MHC binding)}

La propuesta se inspira en los trabajos de \cite{cheng2021bertmhc} y \cite{hashemi2022improved}. Ambos proponen el uso de \textit{transfer  learning} a partir de los modelos pre-entrenados BERT \citep{devlin2018bert} y ESM-1b \citep{rives2021biological} respectivamente. \\


El modelo \textit{Bidirectional Encoder Representations from Transformers.} (BERT), fue diseñado para el pre-entrenamiento de representaciones bidireccionales de textos no etiquetados. Este modelo fue diseñado inicialmente para el procesamiento natural del lenguaje, pero en el trabajo de \cite{rao2019evaluating}, se planteó su uso para secuencias de aminoácidos. Es así que \cite{rao2019evaluating} entrenan BERT con 31 millones de secuencias de proteínas y llaman a su propuesta \textit{Tasks Assessing Protein Embeddings} (TAPE).\\

Recientemente, Facebook desarrolla el modelo ESM-1b \citep{rives2021biological}. La propuesta se basa en el modelo RoBERTa \citep{liu2019roberta}, la cuál es una optimización de BERT. Luego, ESM-1b fue entrenado con la base de datos Uniref50 \citep{suzek2015uniref}, esta base de datos cuenta con aproximadamente 250 millones de secuencias de proteínas. En este caso, se realizó un entrenamiento no supervisado, se ocultaron las etiquetas referentes a la estructura o función de las proteínas.\\

Entonces, la propuesta de la tesis se basa en utilizar \textit{transfer learning} del modelo pre-entrenado ESM-1b, luego se va a utilizar otra red neuronal paralela que se alimente de datos físico-químicos de los aminoácidos. Se propone utilizar las propiedades físico-químicas de los aminoácidos, porque en varios ensayos clínicos se ha comprobado que influyen en la predicción \textit{peptide-MHC binding} y \textit{pMHC-TCR presentation} \citep{gopanenko2020main, borden2022cancer}. Luego, las dos redes neuronales paralelas se unirán en una red neuronal totalmente conectada (ver Figura \ref{fig:proposal}). El objetivo, es aprovechar las propiedades físico-químicas de los aminoácidos para mejorar la afinidad \textit{peptide-MHC}.

Para los entrenamientos y experimentos se utilizará la base de datos HLA3D \citep{li2022hla3d}, esta contiene información de 1296 aminoácidos. Luego, también utilizaremos las muestras recolectadas de \cite{hashemi2022improved}.
\end{comment}

\section{Metodología}

Esta investigación se  enfoca en la tarea de predecir la unión pMHC, descrito en la etapa 3.1 del proceso general para generar vacunas personalizadas basadas en neoantígenos (ver Figura \ref{fig:proposal}). Se ha evaluado  seis modelos Transformers pre-entrenados en diversas tareas de Proteómica como: predicción de estructura de proteínas, predicción de la función de proteínas, etc. Los modelos Transformer son: TAPE \citep{rao2019evaluating}, ProtBert-BFD \citep{elnaggar2021prottrans} y ESM2 \citep{lin2023evolutionary} (ESM2(t6), ESM2(t12), ESM2(t30), ESM2(t33)). Durante la evaluación se realizó \textit{fine-tuning} a los modelos agregando un bloque de BiLSTM al final, de igual forma que lo realizó HLAB \citep{zhang2022hlab}. También se evaluó el uso de Gradient Accumulation Steps (GAS) y el uso de una metodología para congelar las capas del modelo Transformer. En la Figura \ref{fig:proposal}, describimos la propuesta: primero tomamos como entrada el péptido y el MHC, luego estos son concatenados y son recibidos por el modelo Transformer y el bloque BiLSTM respectivamente para predecir su afinidad o unión.




\begin{figure}[H]
	\centering
	\includegraphics[width=\textwidth]{img/proposal/proposal}	
	\caption{Propuesta de \textit{transfer learning} de ESM-1b y una red neuronal paralela para la predicción de la afinidad entre un péptido y MHC (peptide MHC binding).}
	\label{fig:proposal}
\end{figure}


\section{Bases de datos}
Utilizamos secuencias de péptidos del conjunto de datos Anthem \citep{mei2021anthem}. Este conjunto de datos consta de 539,019 muestras para entrenamiento, 179,673 para validación y 172,580 para pruebas. Con más detalle, en la Figura \ref{fig:samples}, presentamos la distribución de las muestras por k-mers; los 9-mers constituyen la mayoría de las muestras en la base de datos.

\begin{figure}[]
	\centering\includegraphics[width=0.8\textwidth]{img/proposal/dataset_samples}
	\caption{
		Cuantificación de las muestras por k-mers dentro de los conjuntos de entrenamiento, validación y pruebas. El conjunto de datos se obtuvo de Anthem \cite{mei2021anthem}.}
	\label{fig:samples}
\end{figure}


\section{Transformer pre-entrenados}

Evaluamos seis modelos de transformadores: TAPE \citep{rao2019evaluating}, ProtBert-BFD \citep{elnaggar2021prottrans} y ESM2 \citep{lin2023evolutionary} (ESM2(t6), ESM2(t12), ESM2(t30), ESM2(t33)). Estos modelos fueron entrenados con grandes conjuntos de datos de secuencias de proteínas como Pfam \citep{el2019pfam},  BFD y UniRef50 \citep{suzek2015uniref}. Además, se realizo \textit{fine-tuning } para la predicción de unión pMHC-I. En la Tabla \ref{tab:pretrained}, presentamos las características de cada modelo.

\begin{table*}[t]%
	\centering
	\caption{Diferencias significativas entre los modelos TAPE, ProtBert-DFB y ESM2. HS: \textit{Hidden size}; AH: \textit{Attention heads}.}
	\label{tab:pretrained}%
	\setlength{\tabcolsep}{0.5em} % for the horizontal padding
	{\renewcommand{\arraystretch}{1.5}% for the vertical padding
	\begin{tabular}{llrrrrr}
		
		\textbf{Model}   & \textbf{BD} & \textbf{Muestras} & \textbf{Capas} & \textbf{HS} & \textbf{AH} & \textbf{Params.} \\
		\midrule
		TAPE             & Pfam             & 30M                   & 12              & 768                  & 12                       & 92M                 \\
		ProtBert-BFD     & BFD              & 2122M                 & 30              & 1024                 & 16                       & 420M                \\
		ESM2(t6)  & Uniref50         & 60M                   & 6               & 320                  & 20                       & 8M                  \\
		ESM2(t12)  & Uniref50         & 60M                   & 12              & 480                  & 20                       & 35M                 \\
		ESM2(t30) & Uniref50         & 60M                   & 30              & 640                  & 20                       & 150M                \\
		ESM2(t33)  & Uniref50         & 60M                   & 33              & 1280                 & 20                       & 650M               \\
		
	\end{tabular}}
	
\end{table*}

\subsection{TAPE}


Tasks Assessing Protein Embeddings (TAPE) \citep{rao2019evaluating} es el primer intento de evaluar el aprendizaje semi-supervisado en secuencias de proteínas. TAPE consta de doce capas de 512 unidades con ocho \textit{attention-heads}, lo que resulta en un total de 92 millones de parámetros. Los autores aplicaron entrenamiento semi-supervisado con la base de datos Pfam \citep{el2019pfam}, que contiene treinta millones de dominios de proteínas. Además, el conjunto de datos Pfam representa un subconjunto del \textit{Knowledge Base UniProt} (UniProtKB) \citep{uniprot2018uniprot}; en particular, Pfam utilizó secuencias de \textit{Reference Proteomes} \citep{finn2016pfam} en lugar de utilizar todo el conjunto de datos de UniProtKB. En consecuencia, Pfam tiene casi la mitad de las secuencias de proteínas que otras bases de datos extraídas de UniProtKB.

\subsection{ProtBert-BFD}

ProtBert-BFD es parte de una familia de modelos de ProtTrans \citep{elnaggar2021prottrans}. Los autores evaluaron varias arquitecturas de aprendizaje profundo con los conjuntos de datos BFD, UniRef50 y UniRef100, cada uno con 2122, 45 y 216 millones de secuencias. Añadido a esto, BFD se considera la colección más extensa de secuencias de proteínas; fusiona UniProt \citep{uniprot2019uniprot} y proteínas de múltiples proyectos de secuenciación de metagenómica. Mientras tanto, UniRef \citep{suzek2015uniref} proporciona un conjunto clusterizado de secuencias de proteínas de UniProtKB. Es importante destacar que el conjunto de datos más grande, BFD, las muestras tienen ruido y contiene errores en las secuencias \citep{elnaggar2021prottrans}.

Algunos de los modelos propuestos son ProtBert-BFD, ProtT5-XL y ProtT5-XXL, que tienen 420 millones, 3 mil millones y 11 mil millones de parámetros, respectivamente. ProtBert-BFD se entrenó con BFD; mientras tanto, los modelos ProtT5 se entrenaron inicialmente con BFD y luego con UniRef50, lo que mejoró el rendimiento en un 2.8\% y un 1.4\% para ProtT5-XL y ProtT5-XXL, respectivamente. Sin embargo, ProtT5-XL superó tanto a ProtBert-BFD como al modelo más grande, ProtT5-XXL. Los autores afirmaron que la cantidad de muestras mejoraba el rendimiento, pero no observaron una similitud consistente con el tamaño del modelo. Sugerían que modelos más grandes ven menos muestras con la misma potencia de cálculo, por lo que los modelos más grandes necesitan conjuntos de datos más grandes. Por esta razón, hemos optado por ProtBert, ya que es más pequeño que ProtT5-XL y creemos que se adapta mejor al tamaño del conjunto de datos actual para esta investigación.

\subsection{ESM2}


ESM-2 \citep{lin2023evolutionary} es una familia de modelos Transformer que tienen  desde 8 millones hasta 15 billones de parámetros. El modelo se basa en BERT \citep{devlin2018bert} y supera a su versión anterior, ESM-1b \citep{rives2021biological}, al eliminar las capas de \textit{dropout} en las capas ocultas y de atención. Además, los autores sugirieron que los métodos de codificación de posición absoluta no se extrapolan bien; en consecuencia, utilizaron la \textit{Rotary Position Embedding} (RoPE). Significativamente, el uso de RoPE aumenta ligeramente el costo de entrenamiento; al mismo tiempo, mejora la calidad del modelo para modelos pequeños \citep{lin2023evolutionary}. Además, los autores utilizaron el conjunto de datos no redundante UniRef50 \citep{suzek2015uniref} de UniProt, que contiene 60 millones de secuencias de proteínas.


\section{Fine-tuning}\label{sec:fine-tuned}
Para realizar \textit{fine-tuning}, apilamos en cascada un bloque BiLSTM al final del modelo pre-entrenado. El BiLSTM se basa en HLAB \citep{zhang2022hlab} y consta de dos capas con 768 unidades. En la Figura \ref{fig:proposal}, presentamos el modelo completo para la predicción de la unión pMHC-I.

Además, está ampliamente establecido que al ajustar modelos de transformadores grandes, las capas finales experimentan cambios más significativos, mientras que las capas iniciales, más cercanas a la entrada, sufren modificaciones relativamente menores \citep{merchant2020happens,lee2019would,kovaleva2019revealing}. En consecuencia, comparamos los resultados de congelar el modelo pre-entrenado y solo actualizar los parámetros de BiLSTM.

Adiconalmente, los modelos de Transformer grandes utilizan bastante memoria de la GPU y generalmente sufren del problema de \textit{vanish gradient}. Por lo tanto, inspirados en trabajos similares sobre entrenamiento de modelos grandes de Transformers para problemas de NLP \citep{anil2021large,zhang2023adam,huang2023measuring}, evaluamos los resultados de aplicar \textit{Gradient Accumulation Steps} durante el entrenamiento.

Finalmente, utilizamos los siguientes hiperparámetros: tasa de aprendizaje de 5e-5, \textit{weight decay} de 0.0001, \textit{momentum} de 0.9, \textit{warn-up steps} de 1000 con \textit{linear decay}, optimizador ADAM ($\beta_1 = 0.9, \beta_2=0.999$) y \textit{early stopping}. Estos valores fueron utilizados por BERTMHC \citep{cheng2021bertmhc} después de buscar los mejores parámetros utilizando \textit{grid search}.

\section{Clasificación binaria y Métricas}

El problema de predicción de unión pMHC es un problema de regresión. Sin embargo, basado en el conjunto de datos utilizado en este estudio, también podría abordarse como un problema de clasificación binaria al seleccionar un umbral apropiado. Luego, las métricas de aprendizaje automático utilizadas en este trabajo son: \textit{accuracy, precision, recall, f-1 score, Area Under the Curve (AUC)}, y \textit{Matthews Correlation Coefficient} (MCC). Todas las métricas están descritas en las ecuaciones siguientes.

\begin{equation}\label{equa:acc}
	Accuracy = \frac{TP+TN}{TP+TN+FP+FN}
\end{equation}

\begin{equation}\label{equa:precision}
	Precision = \frac{TP}{TP+FP}
\end{equation}

\begin{equation}\label{equa:recall}
	Sensitivity = Recall = \frac{TP}{TP+FN}
\end{equation}

\begin{equation}\label{equa:f1}
	F1 = \frac{2*Precision*Recall}{Precision+Recall} = \frac{2 \times TP}{2*TP+FP+FN}
\end{equation}


\begin{equation}\label{equa:FPR}
	Specificity = \frac{TN}{FP+TN}
\end{equation}

\begin{equation}\label{equa:MCC}
	MCC = \frac{TP \times TN - FP \times FN}{ \sqrt{ (TP+FP)(TP  + FN)(TN+FP)(TN+FN)}  }
\end{equation}
 % Introduction
%% ------------------------------------------------------------------- %%
%% ------------------------------------------------------------------- %%
%% ------------------------------------------------------------------- %%
%% ------------------------------------------------------------------- %%
\chapter{Resultados}
\label{cap:resultados}

\lhead{\emph{Resultados}} 

%% ------------------------------------------------------------------- %%
%% ------------------------------------------------------------------- %%
%% ------------------------------------------------------------------- %%
%% ------------------------------------------------------------------- %%
%% ------------------------------------------------------------------- %%
%% ------------------------------------------------------------------- %%

%% -------------------------------------------------------------------- %%
%% -------------------------------------------------------------------- %%



En la Tabla \ref{tab:results}, presentamos el \textit{accuracy, f1 score, precision} y \textit{recall} de cada base de datos (\textit{allele}). Como podemos ver, en todos los casos superamos el 0.9 de \textit{accuracy}, esto valida la propuesta y da origen a seguir trabajando en mejorar la propuesta. \\

Luego, en la Figura \ref{fig:result}, presentamos el \textit{accuracy} obtenido durante el entrenamiento de cada base de datos con el conjunto de muestras de entrenamiento y validación. En este caso, utilizamos el 20\% de las muestras de entrenamiento como validación. Como podemos ver, con solo 10 \textit{epochs},  se lograron buenos resultados. Tambien se evaluao con mas \textit{epochs}, pero los resultados no mejoraron.

\begin{table}[]
	\centering
	\caption{Resultados obtenidos en cada base de datos. }
	\label{tab:results}
	\setlength{\tabcolsep}{0.8em} % for the horizontal padding
	{\renewcommand{\arraystretch}{1.3}% for the vertical padding
		\begin{tabular}{lllll}
			\hline
			\textit{\textbf{Allele}} & \textit{\textbf{Accuracy}} & \textit{\textbf{F1 score}} & \textit{\textbf{Precision}} & \textit{\textbf{Recall}} \\
			\hline
			A*01:01                  & 0.978                      & 0.917                      & 0.982                       & 0.887                    \\
			A*0201                   & 0.962                      & 0.956                      & 0.965                       & 0.948                    \\
			A*02:03                  & 0.992                      & 0.979                      & 0.994                       & 0.969                    \\
			A*31:01                  & 0.980                      & 0.968                      & 0.989                       & 0.951                    \\
			B*44:02                  & 0.991                      & 0.981                      & 0.968                       & 0.997                    \\
			B*44:03                  & 0.992                      & 0.987                      & 0.995                       & 0.980                   
		\end{tabular}
	}
\end{table}

\begin{figure}[H]
	\centering
	\subfigure[A*01:01]{\label{fig:a}\includegraphics[width=0.45\textwidth]{img/neoantigen/acc_A0203}}
	\subfigure[A*02:01]{\label{fig:b}\includegraphics[width=0.45\textwidth]{img/neoantigen/acc_A0201}}
	\subfigure[A*02:03]{\label{fig:a}\includegraphics[width=0.45\textwidth]{img/neoantigen/acc_A0101}}
	\subfigure[A*31:01]{\label{fig:b}\includegraphics[width=0.45\textwidth]{img/neoantigen/acc_A3101}}
	\subfigure[B*44:02]{\label{fig:a}\includegraphics[width=0.45\textwidth]{img/neoantigen/acc_B4402}}
	\subfigure[B*44:03]{\label{fig:b}\includegraphics[width=0.45\textwidth]{img/neoantigen/acc_B4403}}
	
	
	\caption{\textit{Accuracy} durante cada \textit{epoch}, para cada base de datos. Las bases de datos representan las células HLA A*01:01, A*02:01, A*02:03, A*31:01, B*44:02 y B*44:03.}
	\label{fig:result}
\end{figure} % Introduction
%% ------------------------------------------------------------------------- %%
\chapter{Conclusiones}
\label{cap:conclusiones}
\lhead{\emph{Conclusiones}}  


\section{Conclusiones}

En nuestro análisis comparativo de los seis modelos Transformer TAPE, ProtBert-BFD, ESM2(t6), ESM2(t12), ESM2(t30) y ESM2(t33) con la incorporación de GAS y la técnica de congelación de capas, observamos que ESM2(t6)-Freeze y TAPE-GAS lograron los resultados más favorables. Además, observamos que el uso de GAS ofreció una mitigación menor del problema de desvanecimiento del gradiente, lo que permitió el entrenamiento efectivo de modelos Transformer más grandes.

Además, descubrimos que la metodología de congelación de capas aceleró el proceso de entrenamiento y produjo los resultados más favorables para los modelos ESM2. En contraste, el uso de GAS condujo a los mejores resultados para TAPE y ProtBert.

Además, después de entrenar ESM2(t6)-Freeze y TAPE-GAS durante 30 épocas, estos modelos superaron a los métodos de vanguardia, incluidos NetMHCpan4.1, MHCflurry2.0, Anthem, ACME y MixMHCpred2.2, en términos de diversas métricas de rendimiento como el AUC, la precisión, la recuperación, la puntuación f1 y el coeficiente de correlación de Matthews (MCC). Esto demuestra las ventajas de ajustar modelos Transformer grandes para predecir la unión péptido-MHC, subrayando su potencial para mejorar esta tarea crítica.



\section{Trabajos Futuros}
En este trabajo, evaluamos los modelos TAPE, ProtBert-BFD, ESM2(t6), ESM2(t12), ESM2(t30) y ESM2(t33), cada uno con 92, 420, 8, 35, 150 y 650 millones de parámetros respectivamente. Sin embargo, existen otras alternativas como ProtT5-XL y ProtT5-XXL, ESM2(t36) y ESM2(t48), cada uno con 3, 11, 3 y 15 mil millones de parámetros respectivamente. No evaluamos estos modelos debido al tamaño reducido del conjunto de datos y el costo de entrenamiento. No obstante, planeamos evaluar estos enormes modelos Transformer con un conjunto de datos más grande que contenga muestras del conjunto de datos Anthem, MixMHXpred2.2 y la evaluación más reciente de herramientas de predicción de unión pMHC.

Además, dada la considerable inversión de recursos asociada al entrenamiento de modelos Transformer grandes, planeamos investigar las posibles ventajas de utilizar DistilBERT y LoRA para tareas de entrenamiento y predicción futuras.

Además, ajustamos cada modelo Transformer agregando un bloque BiLSTM al final, basado en el trabajo de HLAB. En el futuro, planeamos evaluar la eficacia de un bloque Star-Transformer, similar a la metodología empleada en SMHCpan. Además, considerando los resultados prometedores demostrados en ESM-GAT, creemos que la inclusión de una Red de Atención de Grafos (GAT) podría mejorar significativamente el rendimiento de nuestro modelo en investigaciones futuras. Por último, nos gustaría evaluar la metodología utilizada por TransPHLA, debido a su efectividad demostrada en el manejo de péptidos de diferentes longitudes. % Introduction

%\input{Chapters/Chapter2} % Background Theory 

%\input{Chapters/Chapter3} % Experimental Setup

%\input{Chapters/Chapter4} % Experiment 1

%\input{Chapters/Chapter5} % Experiment 2

%\input{Chapters/Chapter6} % Results and Discussion

%\input{Chapters/Chapter7} % Conclusion



%% ---------------------------------------------------------------- APENDICES ----------------------------------------------------------
%% ----------------------------------------------------------------
% Now begin the Appendices, including them as separate files

%\addtocontents{toc}{\vspace{2em}} % Add a gap in the Contents, for aesthetics

%\appendix % Cue to tell LaTeX that the following 'chapters' are Appendices

%\input{Appendices/AppendixA}	% Appendix Title

%\input{Appendices/AppendixB} % Appendix Title

%\input{Appendices/AppendixC} % Appendix Title

%\addtocontents{toc}{\vspace{2em}}  % Add a gap in the Contents, for aesthetics
%\backmatter

%% ----------------------------------------------------------------
%% ----------------------------------------------------------------
%% --------------------------------------------------------------------------------------------------------------------------


\label{Bibliography}
\lhead{\emph{Bibliografía}}  % Change the left side page header to "Bibliography"
%\bibliographystyle{unsrtnat}  % Use the "unsrtnat" BibTeX style for formatting the Bibliography
\bibliographystyle{apalike}  % Use the "unsrtnat" BibTeX style for formatting the Bibliography
%\bibliography{../Bibliography}  % The references (bibliography) information are stored in the file named "Bibliography.bib"
\bibliography{../bibliography_thesis}  % The references (bibliography) information are stored in the file named "Bibliography.bib"

\end{document}  % The End
%% ----------------------------------------------------------------