\documentclass[10pt]{beamer}
\usepackage[english]{babel}
\usepackage[utf8]{inputenc}
\usepackage[T1]{fontenc}
\usepackage{helvet}
\usepackage{lipsum}  
\usepackage{graphicx,subfigure}
\usepackage{epstopdf}
%-------------------------------------------------------
% INFORMATION IN THE TITLE PAGE
%-------------------------------------------------------

\newcommand{\cstitle}{\textbf{Detección \textit{in Silico} de Neoantígenos Utilizando Transformers y Transfer Learning en el Marco de Desarrollo de Vacunas Personalizadas para Tratar el Cáncer}
\subtitle[]{Tésis de doctorado}}
\newcommand{\cscourseCode}{Detección de Neoantígenos}
\newcommand{\csauthor}{MSc. Vicente Machaca Arceda}
\institute[UNSA]{Universidad Nacional de San Agustín}
\newcommand{\csemail}{vmachaca@utec.edu.pe}
\newcommand{\instituteabr}{UTEC}
\newcommand{\nameUp}{}
\date{2023}
\title[\cscourseCode]{\cstitle}
\author{\csauthor}
%%%%%%%%%%%%%%%%%

%-------------------------------------------------------
% CHOOSE THE THEME
%-------------------------------------------------------
\def\mycmd{0} % UNSA
\def\mycmd{1} % SALLE
%\def\mycmd{2} % UTEC
%-------------------------------------------------------

\if\mycmd0
\usepackage{csformat}
\newcommand{\chref}[3][blue]{\href{#2}{\color{#1}{#3}}}%

\fi

\if\mycmd1
\usetheme[]{Feather}
\newcommand{\chref}[2]{	\href{#1}{{\usebeamercolor[bg]{Feather}#2}} }
\fi

\if\mycmd2
\usetheme{UTEC2020}	
\newcommand{\chref}[3][blue]{\href{#2}{\color{#1}{#3}}}%
\fi

\newcommand{\1}{
	\setbeamertemplate{background}{
		\includegraphics[width=\paperwidth,height=\paperheight]{img/1}
		\tikz[overlay] \fill[fill opacity=0.75,fill=white] (0,0) rectangle (-\paperwidth,\paperheight);
	}
}



%-------------------------------------------------------
% THE BODY OF THE PRESENTATION
%-------------------------------------------------------

\begin{document}
	
	
	\AtBeginSection[]
	{
		\begin{frame}
			\frametitle{Contenido}
			\tableofcontents[currentsection]
		\end{frame}
	}
	
	
	%-------------------------------------------------------
	% THE TITLEPAGE
	%-------------------------------------------------------
	
	\if\mycmd0
	\maketitle
	\fi
	
	\if\mycmd1 % MY THEME
	\1{
		\begin{frame}[plain,noframenumbering] 
			\titlepage 
	\end{frame}}
	\fi
	
	\if\mycmd2
	\begin{frame}
		\titlepage
	\end{frame}
	\fi
	%-------------------------------------------------------
	%-------------------------------------------------------


%-------------------------------------------------------
%-------------------------------------------------------
\begin{frame}{Contenido}
	\tableofcontents
\end{frame}
%-------------------------------------------------------
%-------------------------------------------------------


%%%%%%%%%%%%%%%%%%%%%%%%%%%%%%%%%%%%%%%%%%%%%%%%%%%%%%%%%%%%%%%%%%%%%%%%%%%%%%%%%%%%%%%%%%%%%%%%%%%%%%%%%%%%%%%%
%%%%%%%%%%%%%%%%%%%%%%%%%%%%%%%%%%%%%%%%%%%%%%%%%%%%%%%%%%%%%%%%%%%%%%%%%%%%%%%%%%%%%%%%%%%%%%%%%%%%%%%%%%%%%%%%
%%%%%%%%%%%%%%%%%%%%%%%%%%%%%%%%%%%%%%%%%%%%%%%%%%%%%%%%%%%%%%%%%%%%%%%%%%%%%%%%%%%%%%%%%%%%%%%%%%%%%%%%%%%%%%%%
\section{Contexto y Motivación}
%%%%%%%%%%%%%%%%%%%%%%%%%%%%%%%%%%%%%%%%%%%%%%%%%%%%%%%%%%%%%%%%%%%%%%%%%%%%%%%%%%%%%%%%%%%%%%%%%%%%%%%%%%%%%%%%
%%%%%%%%%%%%%%%%%%%%%%%%%%%%%%%%%%%%%%%%%%%%%%%%%%%%%%%%%%%%%%%%%%%%%%%%%%%%%%%%%%%%%%%%%%%%%%%%%%%%%%%%%%%%%%%%
%%%%%%%%%%%%%%%%%%%%%%%%%%%%%%%%%%%%%%%%%%%%%%%%%%%%%%%%%%%%%%%%%%%%%%%%%%%%%%%%%%%%%%%%%%%%%%%%%%%%%%%%%%%%%%%%

%%%%%%%%%%%%%%%%%%%%%%%%%%%%%%%%%%%%%%%%%%%%%%%%%%%%%%%%%%%%%%%%%%%%%%%%%%%%%%%%%%%%%%%%%%%%%%%%%%%%%%%%%%%%%%%%
\subsection{Estadísticas en Cáncer}
%%%%%%%%%%%%%%%%%%%%%%%%%%%%%%%%%%%%%%%%%%%%%%%%%%%%%%%%%%%%%%%%%%%%%%%%%%%%%%%%%%%%%%%%%%%%%%%%%%%%%%%%%%%%%%%%


%-------------------------------------------------------
%-------------------------------------------------------
\begin{frame}{Contexto y Motivación}{}
	\begin{block}{}
		An la actualidad, el cáncer representa el mayor problema de salud mundial \cite{siegel2023cancer}.
	\end{block}

	\begin{figure}[]
		\centering
		\includegraphics[width=\textwidth]{../img/introduction/cancer_deaths}
		\caption{Ranking de las muertes por cáncer entre 30 y 69 años. \textbf{Fuente}: Atlas Cancer \cite{canceratlas2023}.}
	\end{figure}
\end{frame}
%-------------------------------------------------------
%-------------------------------------------------------

%-------------------------------------------------------
%-------------------------------------------------------
%\begin{frame}{Contexto y Motivación}{Muertes por tipos de cáncer}
%	\begin{figure}[]
%		\centering
%		\includegraphics[width=\textwidth]{../img/introduction/cancer_deaths_woman}
%		\caption{Ranking de las muertes por tipo de cáncer en mujeres. \textbf{Fuente}: The Atlas Cancer \%cite{canceratlas2023}.}
%	\end{figure}
%\end{frame}
%-------------------------------------------------------
%-------------------------------------------------------


%-------------------------------------------------------
%-------------------------------------------------------
%\begin{frame}{Contexto y Motivación}{Muertes por tipos de cáncer}
%	\begin{figure}[]
%		\centering
%		\includegraphics[width=\textwidth]{../img/introduction/cancer_deaths_man}
%		\caption{Ranking de las muertes por tipo de cáncer en hombres. \textbf{Fuente}: The Atlas Cancer \cite{canceratlas2023}.}
%	\end{figure}
%\end{frame}
%-------------------------------------------------------
%-------------------------------------------------------


%-------------------------------------------------------
%-------------------------------------------------------
\begin{frame}{Contexto y Motivación}{Porcentaje de casos y muertes}
	\begin{figure}[]
		\centering
		\includegraphics[width=0.6\textwidth]{../img/introduction/cancer_men_women}
		\caption{Porcentaje de casos y muertes por sexo. \textbf{Fuente} Atlas Cancer \cite{canceratlas2023}.}
	\end{figure}
\end{frame}
%-------------------------------------------------------
%-------------------------------------------------------

%-------------------------------------------------------
%-------------------------------------------------------
\begin{frame}{Contexto y Motivación}{Predicción de nuevos casos}
	\begin{figure}[]
		\centering
		\includegraphics[width=\textwidth]{../img/introduction/cancer_new_cases}
		\caption{Predicción de nuevos casos para el 2040. \textbf{Fuente} Atlas Cancer \cite{canceratlas2023}.}
	\end{figure}
\end{frame}
%-------------------------------------------------------
%-------------------------------------------------------

%%%%%%%%%%%%%%%%%%%%%%%%%%%%%%%%%%%%%%%%%%%%%%%%%%%%%%%%%%%%%%%%%%%%%%%%%%%%%%%%%%%%%%%%%%%%%%%%%%%%%%%%%%%%%%%%
\subsection{Inmunoterapia del Cáncer}
%%%%%%%%%%%%%%%%%%%%%%%%%%%%%%%%%%%%%%%%%%%%%%%%%%%%%%%%%%%%%%%%%%%%%%%%%%%%%%%%%%%%%%%%%%%%%%%%%%%%%%%%%%%%%%%%



%-------------------------------------------------------
%-------------------------------------------------------
\begin{frame}{Contexto y Motivación}{Reacciones distintas para cada paciente}
	%\begin{block}{}
	%	Pacientes con el mismo tipo de cáncer pueden reaccionar de forma disitinta a los mismos tratamientos.
	%\end{block}

	\begin{figure}[]
		\centering
		\includegraphics[width=\textwidth]{../img/introduction/medecine_current}
			\caption{Pacientes con el mismo tipo de cáncer pueden reaccionar de forma disitinta a los mismos tratamientos. \textbf{Fuente} The Atlas Cancer \cite{pdx2023}.}
	\end{figure}
\end{frame}
%-------------------------------------------------------
%-------------------------------------------------------

%-------------------------------------------------------
%-------------------------------------------------------
\begin{frame}{Contexto y Motivación}{Reacciones distintas para cada paciente}

	
	\begin{figure}[]
		\centering
		\includegraphics[width=\textwidth]{../img/introduction/medecine_future}
		\caption{Cada paciente necesita un tratamiento personalizado. \textbf{Fuente} The Atlas Cancer \cite{pdx2023}.}
	\end{figure}
\end{frame}
%-------------------------------------------------------
%-------------------------------------------------------












%-------------------------------------------------------
%-------------------------------------------------------
\begin{frame}{Inmunoterapia del Cáncer}{}		
	Es un tipo de tratamiento contra el Cáncer que estimula las defensas naturales del cuerpo para combatir el Cáncer \cite{inmunoterapy2022}.
		
	\begin{figure}
		\includegraphics[width=0.85\textwidth]{../img/neoantigen/tcell}
		\caption{Ejemplo de como una célula T destruye células del cancer \cite{nortshore2022}.}
	\end{figure}		
\end{frame}
%-------------------------------------------------------
%-------------------------------------------------------


%-------------------------------------------------------
%-------------------------------------------------------
\begin{frame}{Contexto y Motivación}{Inmunoterapia del Cáncer}
	
	
	\begin{figure}[]
		\centering
		\includegraphics[width=0.7\textwidth]{../img/introduction/cancer_immunotherapy}
		\caption{Tipos de tratamientos para la inmunoterapia del cáncer. \textbf{Fuente}: \cite{kciuk2023recent}.}
	\end{figure}
\end{frame}
%-------------------------------------------------------
%-------------------------------------------------------


%%%%%%%%%%%%%%%%%%%%%%%%%%%%%%%%%%%%%%%%%%%%%%%%%%%%%%%%%%%%%%%%%%%%%%%%%%%%%%%%%%%%%%%%%%%%%%%%%%%%%%%%%%%%%%%%
\subsection{Vacunas Personalizadas}
%%%%%%%%%%%%%%%%%%%%%%%%%%%%%%%%%%%%%%%%%%%%%%%%%%%%%%%%%%%%%%%%%%%%%%%%%%%%%%%%%%%%%%%%%%%%%%%%%%%%%%%%%%%%%%%%




%-------------------------------------------------------
%-------------------------------------------------------
\begin{frame}{Contexto y Motivación}{Neoantígenos}
	Es una \textbf{proteína} que se forma en las células de Cáncer cuando ocurre mutaciones en el DNA  \cite{NCIdictionary2022, borden2022cancer}.
	
	\begin{figure}[]
		\centering
		\includegraphics[width=0.7\textwidth]{../img/introduction/neoantigen}
		\caption{Neoantígenos y células T. \textbf{Fuente}: \cite{ucir2023}.}
	\end{figure}
\end{frame}
%-------------------------------------------------------
%-------------------------------------------------------

%-------------------------------------------------------
%-------------------------------------------------------
\begin{frame}{Contexto y Motivación}{Vacunas personalizadas}	
	\begin{figure}[H]
		\centering
		\includegraphics[width=0.9\textwidth]{../img/neoantigen/mhc1.jpg}
		\caption{Presentación de antígenos por MHC-I. Fuente: \cite{zhang2019application}}
		\label{fig:mhc1}
	\end{figure}	
\end{frame}
%-------------------------------------------------------
%-------------------------------------------------------

%%-------------------------------------------------------
%%-------------------------------------------------------
%\begin{frame}{MHC-II}{}		%
%	\begin{figure}[H]
	%		\centering
	%		\includegraphics[width=0.9\textwidth]{../img/neoantigen/mhc2.jpg}
	%		\caption{Presentación de antígenos por MHC-II. Fuente: \cite{zhang2019application}}
	%		\label{fig:mhc2}
	%	\end{figure}	
%\end{frame}
%-------------------------------------------------------
%-------------------------------------------------------

%-------------------------------------------------------
%-------------------------------------------------------
\begin{frame}{Contexto y Motivación}{Vacunas personalizadas}	
	\begin{figure}[]
		\centering
		\includegraphics[width=\textwidth]{../img/introduction/vaccine_1}
		\caption{Proceso para la generación de vacunas contra el cáncer. \textbf{Fuente}: \cite{ucir2023}.}
	\end{figure}
\end{frame}
%-------------------------------------------------------
%-------------------------------------------------------



%-------------------------------------------------------
%-------------------------------------------------------
\begin{frame}{Contexto y Motivación}{Vacunas personalizadas}	
	\begin{figure}[]
		\centering
		\includegraphics[width=\textwidth]{../img/introduction/vaccine_2}
		\caption{Proceso para la generación de vacunas contra el cáncer. \textbf{Fuente}: \cite{ucir2023}.}
	\end{figure}
\end{frame}
%-------------------------------------------------------
%-------------------------------------------------------

%-------------------------------------------------------
%-------------------------------------------------------
\begin{frame}{Contexto y Motivación}{Vacunas personalizadas}	
	\begin{figure}[]
		\centering
		\includegraphics[width=\textwidth]{../img/introduction/vaccine_3}
		\caption{Proceso para la generación de vacunas contra el cáncer. \textbf{Fuente}: \cite{ucir2023}.}
	\end{figure}
\end{frame}
%-------------------------------------------------------
%-------------------------------------------------------

	%-------------------------------------------------------
%-------------------------------------------------------
\begin{frame}{Contexto y Motivación}{Vacunas personalizadas}
	\begin{figure}
		\includegraphics[width=0.5\textwidth]{../img/pipeline/pipeline_spanish}
		\caption{Resumen del proceso de  generación de vacunas contra el cáncer.}
	\end{figure}		
\end{frame}
%-------------------------------------------------------
%-------------------------------------------------------


	%%%%%%%%%%%%%%%%%%%%%%%%%%%%%%%%%%%%%%%%%%%%%%%%%%%%%%%%%%%%%%%%%%%%%%%%%%%%%%%%%%%%%%%%%%%%%%%%%%%%%%%%%%%%%%%%
%%%%%%%%%%%%%%%%%%%%%%%%%%%%%%%%%%%%%%%%%%%%%%%%%%%%%%%%%%%%%%%%%%%%%%%%%%%%%%%%%%%%%%%%%%%%%%%%%%%%%%%%%%%%%%%%
%%%%%%%%%%%%%%%%%%%%%%%%%%%%%%%%%%%%%%%%%%%%%%%%%%%%%%%%%%%%%%%%%%%%%%%%%%%%%%%%%%%%%%%%%%%%%%%%%%%%%%%%%%%%%%%%
\section{Problema y Objetivos}
%%%%%%%%%%%%%%%%%%%%%%%%%%%%%%%%%%%%%%%%%%%%%%%%%%%%%%%%%%%%%%%%%%%%%%%%%%%%%%%%%%%%%%%%%%%%%%%%%%%%%%%%%%%%%%%%
%%%%%%%%%%%%%%%%%%%%%%%%%%%%%%%%%%%%%%%%%%%%%%%%%%%%%%%%%%%%%%%%%%%%%%%%%%%%%%%%%%%%%%%%%%%%%%%%%%%%%%%%%%%%%%%%
%%%%%%%%%%%%%%%%%%%%%%%%%%%%%%%%%%%%%%%%%%%%%%%%%%%%%%%%%%%%%%%%%%%%%%%%%%%%%%%%%%%%%%%%%%%%%%%%%%%%%%%%%%%%%%%%

	%%%%%%%%%%%%%%%%%%%%%%%%%%%%%%%%%%%%%%%%%%%%%%%%%%%%%%%%%%%%%%%%%%%%%%%%%%%%%%%%%%%%%%%%%%%%%%%%%%%%%%%%%%%%%%%%
%%%%%%%%%%%%%%%%%%%%%%%%%%%%%%%%%%%%%%%%%%%%%%%%%%%%%%%%%%%%%%%%%%%%%%%%%%%%%%%%%%%%%%%%%%%%%%%%%%%%%%%%%%%%%%%%
%%%%%%%%%%%%%%%%%%%%%%%%%%%%%%%%%%%%%%%%%%%%%%%%%%%%%%%%%%%%%%%%%%%%%%%%%%%%%%%%%%%%%%%%%%%%%%%%%%%%%%%%%%%%%%%%
\subsection{Problema}
%%%%%%%%%%%%%%%%%%%%%%%%%%%%%%%%%%%%%%%%%%%%%%%%%%%%%%%%%%%%%%%%%%%%%%%%%%%%%%%%%%%%%%%%%%%%%%%%%%%%%%%%%%%%%%%%
%%%%%%%%%%%%%%%%%%%%%%%%%%%%%%%%%%%%%%%%%%%%%%%%%%%%%%%%%%%%%%%%%%%%%%%%%%%%%%%%%%%%%%%%%%%%%%%%%%%%%%%%%%%%%%%%
%%%%%%%%%%%%%%%%%%%%%%%%%%%%%%%%%%%%%%%%%%%%%%%%%%%%%%%%%%%%%%%%%%%%%%%%%%%%%%%%%%%%%%%%%%%%%%%%%%%%%%%%%%%%%%%%

%-------------------------------------------------------
%-------------------------------------------------------
\begin{frame}{Problema}{}	
	\begin{block}{}
		\textbf{Menos del 5\%} de neoantígenos detectados activan el sistema inmune \cite{de2020neoantigen, mill2022neoms, bulik2019deep, bassani2015mass, yadav2014predicting}.
	\end{block}
	
	\pause
	\begin{block}{}
		\begin{itemize} 
			\item La no inclusión en conjunto de varias fuentes de información como DNA-seq, RNA-seq, y datos de MS \cite{kim2018neopepsee}. \pause
			\item  Uso herramientas de bajo desempeño para la predicción del enlace péptido-MHC (pMHC). La mayoría de aplicaciones, se basa en el uso de MHCFlurry \cite{o2020mhcflurry} y NetMHCpan4.1 \cite{reynisson2020netmhcpan}. \pause
			\item No consideran  la predicción del enlace pMHC-TCR  \cite{rubinsteyn2018computational}. \pause
			\item No utilizar información de eventos de \textit{alternative splicing}, variaciones estructurales y  fusión de genes \cite{wood2020neoepiscope}.
		\end{itemize}
	\end{block}
	
\end{frame}
%-------------------------------------------------------
%-------------------------------------------------------

%-------------------------------------------------------
%-------------------------------------------------------
\begin{frame}{Problema}{Formulación del problema}
	\begin{block}{}
		Es un problema de clasificación binaria que toma como entrada la secuencia de aminoácidos de un péptido ($p = \{A, ..., Q\}$) y el MHC ($q = \{A, N, ..., G\}$). Finalmente, necesitamos conocer la probabilidad de afinidad entre $p$ y $q$.
	\end{block}
	
	\begin{figure}
		\includegraphics[width=0.9\textwidth]{../img/neoantigen/problem}
		\caption{Problema de predicción del enlace pMHC.}
	\end{figure}
\end{frame}
%-------------------------------------------------------
%-------------------------------------------------------


%%%%%%%%%%%%%%%%%%%%%%%%%%%%%%%%%%%%%%%%%%%%%%%%%%%%%%%%%%%%%%%%%%%%%%%%%%%%%%%%%%%%%%%%%%%%%%%%%%%%%%%%%%%%%%%%
%%%%%%%%%%%%%%%%%%%%%%%%%%%%%%%%%%%%%%%%%%%%%%%%%%%%%%%%%%%%%%%%%%%%%%%%%%%%%%%%%%%%%%%%%%%%%%%%%%%%%%%%%%%%%%%%
\subsection{Objetivos}
%%%%%%%%%%%%%%%%%%%%%%%%%%%%%%%%%%%%%%%%%%%%%%%%%%%%%%%%%%%%%%%%%%%%%%%%%%%%%%%%%%%%%%%%%%%%%%%%%%%%%%%%%%%%%%%%
%%%%%%%%%%%%%%%%%%%%%%%%%%%%%%%%%%%%%%%%%%%%%%%%%%%%%%%%%%%%%%%%%%%%%%%%%%%%%%%%%%%%%%%%%%%%%%%%%%%%%%%%%%%%%%%%
%%%%%%%%%%%%%%%%%%%%%%%%%%%%%%%%%%%%%%%%%%%%%%%%%%%%%%%%%%%%%%%%%%%%%%%%%%%%%%%%%%%%%%%%%%%%%%%%%%%%%%%%%%%%%%%%

%-------------------------------------------------------
%-------------------------------------------------------
\begin{frame}{Objetivos}{}	
	\begin{block}{Objetivo general}
		Implementar un método \textit{in silico} basado en \textit{Transformers} y \textit{Transfer Learning} para la detección de neoantígenos, enfocados en la predicción de la unión pMHC. 
	\end{block}	

\end{frame}
%-------------------------------------------------------
%-------------------------------------------------------

%-------------------------------------------------------
%-------------------------------------------------------
\begin{frame}{Objetivos}{}	

	\begin{block}{Objetivos específicos}
		\begin{itemize} 
			\item Analizar los métodos que utilizan \textit{Transformers} para la predicción del enlace pMHC en el contexto de detección de neoantígenos. \pause
			\item Analizar los modelos basados en \textit{Transformers} TAPE, ProtBert-BFD, y EMS2 pre-entredados para diversas tareas en Proteómica y de los cuáles se puede aplicar \textit{Transfer Learning}. 	\pause
			\item Implementar \textit{fine-tuning} a los modelos TAPE, ProtBert-BFD, y EMS2 para la tarea de predicción del enlace pMHC, aplicando \textit{Gradient Accumulation Steps} (GAS) y una metodología de congelamiento de capas. \pause
			\item Comparar los modelos de mejor desempeño con las herramientas del estado del arte como: NetMHCpan4.1, MHCFlurry2.0, Anthem, ACME y MixMHCpred2.2.
		\end{itemize}
	\end{block}	
\end{frame}
%-------------------------------------------------------
%-------------------------------------------------------

%%%%%%%%%%%%%%%%%%%%%%%%%%%%%%%%%%%%%%%%%%%%%%%%%%%%%%%%%%%%%%%%%%%%%%%%%%%%%%%%%%%%%%%%%%%%%%%%%%%%%%%%%%%%%%%%
%%%%%%%%%%%%%%%%%%%%%%%%%%%%%%%%%%%%%%%%%%%%%%%%%%%%%%%%%%%%%%%%%%%%%%%%%%%%%%%%%%%%%%
\section{Estado del arte}
%%%%%%%%%%%%%%%%%%%%%%%%%%%%%%%%%%%%%%%%%%%%%%%%%%%%%%%%%%%%%%%%%%%%%%%%%%%%%%%%%%%%%%%%%%%%%%%%%%%%%%%%%%%%%%%%
%%%%%%%%%%%%%%%%%%%%%%%%%%%%%%%%%%%%%%%%%%%%%%%%%%%%%%%%%%%%%%%%%%%%%%%%%%%%%%%%%%%%%%

%-------------------------------------------------------
%-------------------------------------------------------
\begin{frame}{Estado del arte}{}	
	
\end{frame}
%-------------------------------------------------------
%-------------------------------------------------------


%%%%%%%%%%%%%%%%%%%%%%%%%%%%%%%%%%%%%%%%%%%%%%%%%%%%%%%%%%%%%%%%%%%%%%%%%%%%%%%%%%%%%%%%%%%%%%%%%%%%%%%%%%%%%%%%
%%%%%%%%%%%%%%%%%%%%%%%%%%%%%%%%%%%%%%%%%%%%%%%%%%%%%%%%%%%%%%%%%%%%%%%%%%%%%%%%%%%%%%
\section{Propuesta}
%%%%%%%%%%%%%%%%%%%%%%%%%%%%%%%%%%%%%%%%%%%%%%%%%%%%%%%%%%%%%%%%%%%%%%%%%%%%%%%%%%%%%%%%%%%%%%%%%%%%%%%%%%%%%%%%
%%%%%%%%%%%%%%%%%%%%%%%%%%%%%%%%%%%%%%%%%%%%%%%%%%%%%%%%%%%%%%%%%%%%%%%%%%%%%%%%%%%%%%

%-------------------------------------------------------
%-------------------------------------------------------
\begin{frame}{Propuesta}{}

	\begin{figure}[H]
		\centering
		\includegraphics[width=\textwidth]{../img/proposal/proposal}	
		\caption{Propuesta para la predicción del enlace pMHC.}
		\label{fig:neo_det_seq}
	\end{figure}
\end{frame}
%-------------------------------------------------------
%-------------------------------------------------------


%-------------------------------------------------------
%-------------------------------------------------------
\begin{frame}{Propuesta}{\textit{Fine-tuning}}
	
\end{frame}
%-------------------------------------------------------
%-------------------------------------------------------

%-------------------------------------------------------
%-------------------------------------------------------
\begin{frame}{Propuesta}{\textit{Gradient Accumulation Steps}}
	
\end{frame}
%-------------------------------------------------------
%-------------------------------------------------------

%-------------------------------------------------------
%-------------------------------------------------------
\begin{frame}{Propuesta}{\textit{Congelamiento de capas}}
	
\end{frame}
%-------------------------------------------------------
%-------------------------------------------------------



%%%%%%%%%%%%%%%%%%%%%%%%%%%%%%%%%%%%%%%%%%%%%%%%%%%%%%%%%%%%%%%%%%%%%%%%%%%%%%%%%%%%%%%%%%%%%%%%%%%%%%%%%%%%%%%%
%%%%%%%%%%%%%%%%%%%%%%%%%%%%%%%%%%%%%%%%%%%%%%%%%%%%%%%%%%%%%%%%%%%%%%%%%%%%%%%%%%%%%%
\section{Experimentos y Resultados}
%%%%%%%%%%%%%%%%%%%%%%%%%%%%%%%%%%%%%%%%%%%%%%%%%%%%%%%%%%%%%%%%%%%%%%%%%%%%%%%%%%%%%%%%%%%%%%%%%%%%%%%%%%%%%%%%
%%%%%%%%%%%%%%%%%%%%%%%%%%%%%%%%%%%%%%%%%%%%%%%%%%%%%%%%%%%%%%%%%%%%%%%%%%%%%%%%%%%%%%

%%%%%%%%%%%%%%%%%%%%%%%%%%%%%%%%%%%%%%%%%%%%%%%%%%%%%%%%%%%%%%%%%%%%%%%%%%%%%%%%%%%%%%%%%%%%%%%%%%%%%%%%%%%%%%%%
%%%%%%%%%%%%%%%%%%%%%%%%%%%%%%%%%%%%%%%%%%%%%%%%%%%%%%%%%%%%%%%%%%%%%%%%%%%%%%%%%%%%%%
\subsection{Base de datos}
%%%%%%%%%%%%%%%%%%%%%%%%%%%%%%%%%%%%%%%%%%%%%%%%%%%%%%%%%%%%%%%%%%%%%%%%%%%%%%%%%%%%%%%%%%%%%%%%%%%%%%%%%%%%%%%%
%%%%%%%%%%%%%%%%%%%%%%%%%%%%%%%%%%%%%%%%%%%%%%%%%%%%%%%%%%%%%%%%%%%%%%%%%%%%%%%%%%%%%%

%-------------------------------------------------------
%-------------------------------------------------------
\begin{frame}{Base de datos}{}
	
	\textit{Training}: 539,019; \textit{Validation}: 179,673; y \textit{Testing}: 172,580.
	
	\begin{figure}[]
		\centering\includegraphics[width=0.7\textwidth]{../img/proposal/dataset_samples}
		\caption{
			Número de muestras por  \textit{k-mer}.}
		\label{fig:samples}
	\end{figure}
	
\end{frame}
%-------------------------------------------------------
%-------------------------------------------------------

%%%%%%%%%%%%%%%%%%%%%%%%%%%%%%%%%%%%%%%%%%%%%%%%%%%%%%%%%%%%%%%%%%%%%%%%%%%%%%%%%%%%%%%%%%%%%%%%%%%%%%%%%%%%%%%%
%%%%%%%%%%%%%%%%%%%%%%%%%%%%%%%%%%%%%%%%%%%%%%%%%%%%%%%%%%%%%%%%%%%%%%%%%%%%%%%%%%%%%%
\subsection{Modelos pre-entrenados}
%%%%%%%%%%%%%%%%%%%%%%%%%%%%%%%%%%%%%%%%%%%%%%%%%%%%%%%%%%%%%%%%%%%%%%%%%%%%%%%%%%%%%%%%%%%%%%%%%%%%%%%%%%%%%%%%
%%%%%%%%%%%%%%%%%%%%%%%%%%%%%%%%%%%%%%%%%%%%%%%%%%%%%%%%%%%%%%%%%%%%%%%%%%%%%%%%%%%%%%	

%-------------------------------------------------------
%-------------------------------------------------------
\begin{frame}{Modelos pre-entrenados}{}
	
	\begin{table}
		\centering
		\caption{Diferencias entre TAPE, ProtBert-DFB, y ESM2. HS: \textit{Hidden size}; AH: \textit{Attention heads}.}
		\label{tab:pretrained}%
		\setlength{\tabcolsep}{0.5em} % for the horizontal padding
		{\renewcommand{\arraystretch}{1.5}% for the vertical padding
			\footnotesize
			\begin{tabular}{llrrrrr}
				
				\textbf{Modelo}   & \textbf{BD} & \textbf{Muestras} & \textbf{Capas} & \textbf{HS} & \textbf{AH} & \textbf{Params.} \\
				\hline
				TAPE             & Pfam             & 30M                   & 12              & 768                  & 12                       & 92M                 \\
				ProtBert-BFD     & BFD              & 2122M                 & 30              & 1024                 & 16                       & 420M                \\
				ESM2(t6)  & Uniref50         & 60M                   & 6               & 320                  & 20                       & 8M                  \\
				ESM2(t12)  & Uniref50         & 60M                   & 12              & 480                  & 20                       & 35M                 \\
				ESM2(t30) & Uniref50         & 60M                   & 30              & 640                  & 20                       & 150M                \\
				ESM2(t33)  & Uniref50         & 60M                   & 33              & 1280                 & 20                       & 650M               \\
				
		\end{tabular}}
		
	\end{table}
	
\end{frame}
%-------------------------------------------------------
%-------------------------------------------------------


%%%%%%%%%%%%%%%%%%%%%%%%%%%%%%%%%%%%%%%%%%%%%%%%%%%%%%%%%%%%%%%%%%%%%%%%%%%%%%%%%%%%%%%%%%%%%%%%%%%%%%%%%%%%%%%%
%%%%%%%%%%%%%%%%%%%%%%%%%%%%%%%%%%%%%%%%%%%%%%%%%%%%%%%%%%%%%%%%%%%%%%%%%%%%%%%%%%%%%%
\subsection{Resultados}
%%%%%%%%%%%%%%%%%%%%%%%%%%%%%%%%%%%%%%%%%%%%%%%%%%%%%%%%%%%%%%%%%%%%%%%%%%%%%%%%%%%%%%%%%%%%%%%%%%%%%%%%%%%%%%%%
%%%%%%%%%%%%%%%%%%%%%%%%%%%%%%%%%%%%%%%%%%%%%%%%%%%%%%%%%%%%%%%%%%%%%%%%%%%%%%%%%%%%%%	

%-------------------------------------------------------
%-------------------------------------------------------
\begin{frame}{Resultados}{Entrenamiento por 3 \textit{epochs}}
	\begin{figure}
		\centering		
		\includegraphics[width=0.7\textwidth]{../img/results/metrics_comparion_by_model.png}			
		\caption{Comparación del AUC por modelo y metodología de entrenamiento.}
	\end{figure}
\end{frame}
%-------------------------------------------------------
%-------------------------------------------------------


%-------------------------------------------------------
%-------------------------------------------------------
\begin{frame}{Resultados}{Problema de \textit{vanish gradient} para ESM2(t6)}
	\centering
	\includegraphics[width=\textwidth]{../img/results/t6_epoch0}
	\includegraphics[width=\textwidth]{../img/results/t6_epoch3}
\end{frame}
%-------------------------------------------------------
%-------------------------------------------------------

%-------------------------------------------------------
%-------------------------------------------------------
\begin{frame}{Resultados}{Problema de \textit{vanish gradient} para ESM2(t30)}
	\centering

	\includegraphics[width=\textwidth]{../img/results/t30_epoch0}
	\includegraphics[width=\textwidth]{../img/results/t30_epoch3}
\end{frame}
%-------------------------------------------------------
%-------------------------------------------------------

%-------------------------------------------------------
%-------------------------------------------------------
\begin{frame}{Resultados}{Entrenamiento por 30 \textit{epochs}}
	\begin{table}[]
		\centering
		\setlength{\tabcolsep}{0.5em} % for the horizontal padding
		{\renewcommand{\arraystretch}{1.5}% for the vertical padding
			\scriptsize
			\begin{tabular}{lllllll} 
				\textbf{}            & \textbf{Accuracy} & \textbf{Precision} & \textbf{Recall} & \textbf{F1-score} & \textbf{AUC}    & \textbf{MCC}    \\ \hline
				ESM2(t6)-Normal             & 0.9390            & 0.9333             & \textbf{0.9453} & 0.9392            & 0.9797          & 0.8780                    \\
				ESM2(t6)-Freeze      & \textbf{0.9401}   & \textbf{0.9398}    & 0.9402          & \textbf{0.9400}   & \textbf{0.9830} & \textbf{0.8802}             \\
				ESM2(t6)-GAS         & 0.9366            & 0.9322             & 0.9413          & 0.9368            & 0.9818          & 0.8732                     \\
				ESM2(t6)-Freeze-GAS  & 0.9354            & 0.9326             & 0.9383          & 0.9355            & 0.9813          & 0.8708                    \\ \hline
				ESM2(t30)-Normal            & -                 & -                  & -               & -                 & -               & -                                 \\
				ESM2(t30)-Freeze     & \textbf{0.9393}   & 0.9304             & \textbf{0.9493} & \textbf{0.9397}   & 0.9787          & \textbf{0.8787}           \\
				ESM2(t30)-GAS        & 0.9346            & \textbf{0.9337}    & 0.9352          & 0.9345            & 0.9808          & 0.8691                    \\
				ESM2(t30)-Freeze-GAS & 0.9363            & 0.9319             & 0.9411          & 0.9365            & \textbf{0.9818} & 0.8726                    \\ \hline
				TAPE-Normal                 & -                 & -                  & -               & -                 & -               & -                                  \\
				TAPE-Freeze          & 0.9395            & \textbf{0.9404}    & 0.9382          & 0.9393            & 0.9815          & 0.8790                      \\
				TAPE-GAS             & \textbf{0.9415}   & 0.9352             & \textbf{0.9484} & \textbf{0.9418}   & \textbf{0.9841} & \textbf{0.8831}            \\
				TAPE-Freeze-GAS      & 0.9359            & 0.9297             & 0.9428          & 0.9362            & 0.9820          & 0.8719               \\     
		\end{tabular}}
	\end{table}
\end{frame}
%-------------------------------------------------------
%-------------------------------------------------------


%-------------------------------------------------------
%-------------------------------------------------------
\begin{frame}{Resultados}{Comparación con los métodos \textit{state-of-art}}
	\begin{figure}
		\includegraphics[width=0.75\textwidth]{../img/results/metrics_comparison}
		\caption{Comparación de TAPE-GAS y ESM2(t6) contra los mejores métodos del estado del arte.}
	\end{figure}
\end{frame}
%-------------------------------------------------------
%-------------------------------------------------------

%-------------------------------------------------------
%-------------------------------------------------------
\begin{frame}{Resultados}{Comparación con los métodos \textit{state-of-art}}
	\begin{figure}
		\includegraphics[width=0.75\textwidth]{../img/results/ROC_comparison}
		\caption{Comparación de TAPE-GAS y ESM2(t6) contra los mejores métodos del estado del arte.}
	\end{figure}
\end{frame}
%-------------------------------------------------------
%-------------------------------------------------------

%-------------------------------------------------------
%-------------------------------------------------------
\begin{frame}{Resultados}{Comparación con los métodos \textit{state-of-art}}
	\begin{table}[]
		\centering
		\caption{Desempeño de TAPE-GAS y ESM2(t6)-Freeze, entrenados por 30 \textit{epochs}, contra Anthem, NetMHCpan4.1, ACME, MixMHCpred2.2, y MhcFlurry2.0.}
		\setlength{\tabcolsep}{0.5em} % for the horizontal padding
		{\renewcommand{\arraystretch}{1.5}% for the vertical padding
			\scriptsize
			\begin{tabular}{lllllll} 
				& \textbf{Accuracy} & \textbf{Precision} & \textbf{Recall} & \textbf{F1-score} & \textbf{AUC}    & \textbf{MCC}    \\ \hline
				TAPE-GAS        & \textbf{0.9415}   & 0.9352             & \textbf{0.9484} & \textbf{0.9418}   & \textbf{0.9841} & \textbf{0.8831} \\
				ESM2(t6)-Freeze & \textbf{0.9401}   & 0.9398             & \textbf{0.9402} & \textbf{0.9400}   & \textbf{0.9830} & \textbf{0.8802} \\
				
				Anthem          & 0.8811            & \textbf{0.9786}    & 0.7787          & 0.8673            & 0.9768          & 0.7785          \\
				NetMHCpan4.1    & 0.8312            & \textbf{0.9844}    & 0.6724          & 0.7991            & 0.9557          & 0.6982          \\
				
				ACME            & 0.8452            & 0.9717             & 0.7105          & 0.8208            & 0.9476          & 0.7165          \\
				MixMHCpred2.2   & 0.8857            & 0.9155             & 0.8493          & 0.8811            & 0.9386          & 0.7733          \\
				MhcFlurry2.0    & 0.9093            & 0.9211             & 0.8948          & 0.9078            & 0.9642          & 0.8189 \\         
		\end{tabular}}
	\end{table}
\end{frame}
%-------------------------------------------------------
%-------------------------------------------------------


%%%%%%%%%%%%%%%%%%%%%%%%%%%%%%%%%%%%%%%%%%%%%%%%%%%%%%%%%%%%%%%%%%%%%%%%%%%%%%%%%%%%%%%%%%%%%%%%%%%%%%%%%%%%%%%%
%%%%%%%%%%%%%%%%%%%%%%%%%%%%%%%%%%%%%%%%%%%%%%%%%%%%%%%%%%%%%%%%%%%%%%%%%%%%%%%%%%%%%%
\section{Discusión y Conclusiones}
%%%%%%%%%%%%%%%%%%%%%%%%%%%%%%%%%%%%%%%%%%%%%%%%%%%%%%%%%%%%%%%%%%%%%%%%%%%%%%%%%%%%%%%%%%%%%%%%%%%%%%%%%%%%%%%%
%%%%%%%%%%%%%%%%%%%%%%%%%%%%%%%%%%%%%%%%%%%%%%%%%%%%%%%%%%%%%%%%%%%%%%%%%%%%%%%%%%%%%%




%-------------------------------------------------------
%-------------------------------------------------------
\begin{frame}{Discusión}{¿Porque el modelo mas pequeño de la familia ESM2 es el mejor?}
	
	\begin{block}{}
		El modelo más pequeño, ESM2(t6) supero a los demas. Las causas de este fenómeno pueden ser: 
	
	\end{block}
	
	\pause
	
	\begin{block}{}
		\begin{itemize}
			\item En conjunto de datos que consta de 559,019 muestras, que no consideramos lo suficientemente grande para ESM2(t33), un modelo que cuenta con 650 millones de parámetros. \pause
			
			\item El uso de \textit{Rotary Position Embedding} (RoPE) en lugar de la codificación posicional absoluta. Si bien RoPE puede llevar a un ligero aumento en el costo de entrenamiento, se ha observado que mejora la calidad de los resultados, especialmente para modelos más pequeños \cite{lin2023evolutionary}.
		\end{itemize}
	\end{block}
	
\end{frame}
%-------------------------------------------------------
%-------------------------------------------------------

%-------------------------------------------------------
%-------------------------------------------------------
\begin{frame}{Discusión}{Congelamiento de capas y GAS}
	
	
	\begin{block}{Congelamiento de capas}
	Para los modelos ESM2, esta metodología arrojó los mejores resultados, mientras que para TAPE y ProtBert-BFD, produjo los resultados esperados
	\end{block}

\pause
	
	\begin{block}{GAS}
	Esta técnica alivia ligeramente el problema de \textit{vanish gradients} acumulando las gradientes durante las iteraciones. En si, este técnica extiende la cantidad de iteraciones de entrenamiento que se pueden realizar antes de que el modelo posiblemente vuelva a enfrentar el problema de \textit{vanish gradient}.
	\end{block}
\end{frame}
%-------------------------------------------------------
%-------------------------------------------------------


%-------------------------------------------------------
%-------------------------------------------------------
\begin{frame}{Discusión}{TAPE, ProtBert-BFD and ESM2}
	
	\begin{block}{ProtBert-BFD}
		ProtBert-BFD (420M parámetros) obtuvo el peor resultado a pesar de que este modelo fue \textbf{pre-entrenado con el conjunto de datos más grande (2122 millones de muestras)}. Las causas son: 
	\end{block}

	\begin{block}{}
		\begin{itemize}
			\item  Ruido en las muestras y a los errores en las secuencias en el conjunto de datos BFD \cite{elnaggar2021prottrans}
			\item Los modelos \textit{Transformer} grandes requieren más datos para el entrenamiento \cite{elnaggar2021prottrans}, y en nuestro caso este modelo se entreno con 559,019 muestras.
		\end{itemize}
	\end{block}
	
\end{frame}
%-------------------------------------------------------
%-------------------------------------------------------



%-------------------------------------------------------
%-------------------------------------------------------
\begin{frame}{Discusión}{TAPE, ProtBert-BFD and ESM2}
	
	\begin{block}{TAPE}
		TAPE logró los mejores resultados. Este solo fue pre-entrenado con 30 millones de muestras; sin embargo, secuencias pertenecen a \textit{Reference Proteomes} en lugar de abarcar toda la base de datos de UniProtKB \cite{finn2016pfam}. 
	\end{block}
	
	\begin{block}{ESM2}
		ESM2(t6) logró resultados que compiten estrechamente con el desempeño de TAPE; sin embargo ESM2(t6) tiene 8M versus los 92M de parámetros de TAPE.		
	\end{block}
\end{frame}
%-------------------------------------------------------
%-------------------------------------------------------

%-------------------------------------------------------
%-------------------------------------------------------
\begin{frame}[allowframebreaks]
	\frametitle{References}
	%\bibliographystyle{amsalpha}
	\bibliographystyle{IEEEtran}
	\bibliography{../bibliography_thesis.bib}
\end{frame}
%-------------------------------------------------------
%-------------------------------------------------------

%-------------------------------------------------------
%-------------------------------------------------------
\if\mycmd1 % MY THEME
\1{
	{\1
		\begin{frame}[plain,noframenumbering]
			%\finalpage{Thank you}
			\begin{figure}[]
				\centering
				\includegraphics[width=\textwidth,height=0.7\textheight,keepaspectratio]{img/question.png}
				%\label{img:mot2}
				%\caption{Image example in 2 gray levels.}
			\end{figure}
	\end{frame}}
	\else % CS THEME
	\begin{frame}{Questions?}
		\begin{figure}[]
			\centering
			\includegraphics[width=\textwidth,height=0.7\textheight,keepaspectratio]{img/question.png}
			%\label{img:mot2}
			%\caption{Image example in 2 gray levels.}
		\end{figure}
		
	\end{frame}
	\fi
	%-------------------------------------------------------
	%-------------------------------------------------------
	

\end{document}