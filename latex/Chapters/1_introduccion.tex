%% ------------------------------------------------------------------- %%
%% ------------------------------------------------------------------- %%
\chapter{Introducción}
\label{cap:introduccion}
\lhead{\emph{Introducción}}  % Change the left side page header to "Bibliography"


%% ------------------------------------------------------------------- %%
\section{Motivación}
\label{sec:motivacion}
El cáncer representa el mayor problema de salud mundial \citep{siegel2022cancer} y es el causante líder de muertes, solo en el 2020 se registraron alrededor de 10 millones de muertes y aproximadamente cada año 400000 niños desarrollan cáncer \citep{whocancer2022}. Lamentablemente, a pesar de muchos esfuerzos por mitigar las muertes causadas por esta enfermedad, los métodos tradicionales basados en cirugías, radioterapias y quimioterapias tienen baja efectividad \citep{peng2019neoantigen}. En este contexto, surge el desarrollo de la inmunoterapia del cáncer, el cuál tiene el objetivo estimular el sistema inmune de un paciente. La idea es que nuestro propio sistema inmune sea capaz de reconocer las células de cáncer como agentes extraños y por consiguiente elimine dichas células. Existen varios enfoques y metodologías en la inmunoterapia del cáncer, de estos, la de mayor estudio y efectividad es el desarrollo de vacunas personalizadas \citep{borden2022cancer}.

El desarrollo de vacunas personalizadas contra el cáncer es un proceso largo y depende de una correcta detección de neo antígenos. Estos neo antígenos son péptidos\footnote{Secuencias cortas de aminoacidos.} que solo se presentan en células cancerosas; entonces, el objetivo es entrenar a los linfocitos (células T) de un paciente para que estos puedan reconocer los neo antígenos y asi activar el sistema inmune.

Determinar qué estrategia o método de detección de neo antígenos es el adecuado o en qué circunstancias conviene la aplicación de alguno, es muy importante para el desarrollo de vacunas personalizadas \citep{de2020neoantigen, peng2019neoantigen}.  Sin embargo, a pesar de los esfuerzos de los investigadores en desarrollar métodos y herramientas,  menos del 3\% de los neo antígenos detectados logran activar a las células T (sistema inmune) \citep{de2020neoantigen}. De esta forma, es relevante que se continue con la investigación y desarrollo de nuevos métodos que permitan detectar neo antígenos.


\section{Problema}
\label{sec:problema}

Los neo antígenos son peṕtidos mutados específicos de tumores y son considerados los principales causantes de una respuesta inmune \citep{borden2022cancer, chen2021challenges, gopanenko2020main}. Es así que surgen varios esfuerzos e investigación en la Inmunoterapia del cáncer, concentradas en el estudio y detección de neo antígenos. En la actualidad existen tres clases de tratamientos basados en la representación y expresión de neo antígenos: vacunas personalizadas, terapias adoptivas de células T y \textit{immune checkpoint inhibitors}. De los métodos mencionados anteriormente, el desarrollo de vacunas personalizadas es considerado uno de los métodos con mayor probabilidad de éxito \citep{borden2022cancer}. Incluso varias compañías como BioNTech, Genocea Biosciences, Neon Therapeutics y Gritstone Oncology realizan investigación y ofrecen el servicio de generar vacunas personalizadas a pacientes de cáncer.

Según lo mencionado anteriormente, la detección de neo antígenos es un factor clave en el desarrollo de vacunas personalizadas. En este proceso el compuesto \textit{Major Histocompatibility Complex} (MHC), juega un papel muy importante, es el encargado de presentar los péptidos a la células T \citep{hashemi2022improved}. Para el caso de células humanas el gen MHC es conocido como Human Leukocyte Antigens (HLA) y es polimórfico, se cree que existen las 10000 diferentes \textit{HLA-I alleles} \citep{abelin2017mass}, esto complica mucho más la detección de neo antígenos. 

El ciclo de vida de un neo antígeno para células con núcleo podría resumirse como: primero una proteína es degradada en péptidos en el citoplasma de las células, luego los péptidos se enlazan a la molecula MHC (\textit{pMHC binding}), luego este compuesto sigue un trayecto hasta llegar a la membrana de la célula (\textit{pMHC presentation}), finalmente el compuesto pMHC es reconocido por  el T-cell Receptor (TCR) de las células T y así si activaría el sistema inmune. Además, el número de posibles péptidos enlazables a MHC  son entre 1000 a 10000, esto es el 0.1\% de los posibles péptidos  de 9 aminoacidos\footnote{La mayoría de péptidos enlazados a moléculas MHC-I tienen 9 aminoácidos, se suele utilizar el termino \textit{n-mer} para referirse a péptidos de \textit{n} aminoácidos.} \citep{abelin2017mass}. En este proceso, el objetivo es detectar los péptidos (neo antígenos) que llegan a la membrana de la célula, luego con ayuda de procedimientos de biotecnología, se entrena a las células T de un paciente para que aprenda a reconocer los neo antígenos.


El problema de \textit{pMHC binding} está casi solucionado con una precisión de 0.98 por parte de la herramienta NetMHCPan 4.1 \citep{reynisson2020netmhcpan}. Sin embargo, no es bueno limitar la detección de neo antígenos solo al problema de \textit{pMHC binding}, porque la mayoría de estos compuestos no llegan a la membrana \citep{mill2022neoms}, a este problema se le conoce como \textit{pMHC presentation}. Por ejemplo, se sabe que menos del 5\% de péptidos detectados llegan a la membrana \citep{de2020neoantigen, mill2022neoms, bulik2019deep, bassani2015mass, yadav2014predicting}. Además, existen herramientas como NeyMHC, NetMHCpan y MHCFlurry que tienen un buen desemepeño en \textit{pMHC binding}, pero con resultados pobres en  \textit{pMHC presentation} \citep{bulik2019deep}.



\subsection{Formulación del problema}

Menos del 5\% de péptidos detectados en \textit{pMHC binding}, llegan a la membrana de la células, para que luego sean reconocidos por las células T.  El proceso por el cúal un péptido enlazado a MHC llegue a la membrana es conocido como  \textit{pMHC presentation}, pero en este problema las propuestas recientes solo llegan a un 0.61 de precisión y 0.4 de \textit{recall}. En este contexto, la tesis se enfoca en el problema de \textit{pMHC presentation}, considerándolo como un problema de clasificación binaria, y tomando como entrada la secuencia de aminoácidos del péptido y la secuencia de aminoácidos de la proteína MHC. 



%% ------------------------------------------------------------------- %%
\section{Objetivos}
\label{sec:objetivos}

\subsection{Objetivo General}

Proponer un método basado en \textit{deep learning} para la detección de neo antígenos, enfocados en el problema de \textit{pMHC presentation}. 

\subsection{Objetivos específicos}

\begin{enumerate}[(a)]
\item Realizar una revisión sistemática de la literatura e implementar los métodos con mejor desempeño en la detección de neo antígenos.
\item Proponer e implementar un método basado en \textit{deep learning} para la detección de neo antígenos.		
\item Evaluar el método propuesto en bases de datos publicas.
\end{enumerate}

%% ------------------------------------------------------------------- %%
\section{Contribuciones}
\label{sec:contribuciones}
Las principales contribuciones de este trabajo son:

\begin{enumerate}[(a)]
	\item Una .........................;
	\item Una .........................;
	\item Una .........................;
	
\end{enumerate}

%% ------------------------------------------------------------------- %%
\section{Organización del Trabajo}
\label{sec:organizaciondeltrabajo}
En el Capítulo~\ref{cap:marcoteorico} se presentan los conceptos básicos ....

en el Capítulo~\ref{cap:estadodelarte} se describen los trabajos relacionados a la presente tesis.

..............................

Finalmente, en el Capítulo~\ref{cap:conclusiones} son expuestos las conclusiones del presente trabajo así como 
tambien las direcciones para continuar con el mismo en la sección de trabajos futuros.

%Em anexos está uma exemplificação da especificação da arquitetura, 
%neste caso é descrita a especificação da arquitetura 
%DAS-3 (Apêndice \ref{ape:especificacoes}).


%%%%%
