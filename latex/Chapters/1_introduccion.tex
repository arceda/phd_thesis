%% ------------------------------------------------------------------- %%
%% ------------------------------------------------------------------- %%
\chapter{Introducción}
\label{cap:introduccion}
\lhead{\emph{Introducción}}  % Change the left side page header to "Bibliography"


%% ------------------------------------------------------------------- %%
\section{Motivación}
\label{sec:motivacion}
El cáncer representa el mayor problema de salud mundial \citep{siegel2022cancer} y es el causante líder de muertes, solo en el 2020 se registraron alrededor de 10 millones de muertes y aproximadamente cada año 400000 niños desarrollan cáncer \citep{whocancer2022}. Lamentablemente, a pesar de muchos esfuerzos por mitigar las muertes causadas por esta enfermedad, los métodos tradicionales basados en cirugías, radioterapias y quimioterapias tienen baja efectividad \citep{peng2019neoantigen}. En este contexto, surge el desarrollo de la inmunoterapia del cáncer, el cuál tiene el objetivo estimular el sistema inmune de un paciente. La idea es que nuestro propio sistema inmune sea capaz de reconocer las células de cáncer como agentes extraños y por consiguiente elimine dichas células. Existen varios enfoques y metodologías en la inmunoterapia del cáncer, de estos, la de mayor estudio y efectividad es el desarrollo de vacunas personalizadas \citep{borden2022cancer}.

El desarrollo de vacunas personalizadas contra el cáncer es un proceso largo y depende de una correcta detección de neo antígenos. Estos neo antígenos son péptidos\footnote{Secuencias cortas de aminoacidos.} que solo se presentan en células cancerosas; entonces, el objetivo es entrenar a los linfocitos (células T) de un paciente para que estos puedan reconocer los neo antígenos y asi activar el sistema inmune.

Determinar qué estrategia o método de detección de neo antígenos es el adecuado o en qué circunstancias conviene la aplicación de alguno, es muy importante para el desarrollo de vacunas personalizadas \citep{de2020neoantigen, peng2019neoantigen}.  Sin embargo, a pesar de los esfuerzos de los investigadores en desarrollar métodos y herramientas,  menos del 3\% de los neo antígenos detectados logran activar a las células T (sistema inmune) \citep{de2020neoantigen}. De esta forma, es relevante que se continue con la investigación y desarrollo de nuevos métodos que permitan detectar neo antígenos.


\section{Problema}
\label{sec:problema}

Los neo antígenos son peṕtidos mutados específicos de tumores y son considerados los principales causantes de una respuesta inmune \citep{borden2022cancer, chen2021challenges, gopanenko2020main}. Es así que surgen varios esfuerzos e investigación en la Inmunoterapia del cáncer, concentradas en el estudio y detección de neo antígenos. En la actualidad existen tres clases de tratamientos basados en la representación y expresión de neo antígenos: vacunas personalizadas, terapias adoptivas de células T y \textit{immune checkpoint inhibitors}. De los métodos mencionados anteriormente, el desarrollo de vacunas personalizadas es considerado uno de los métodos con mayor probabilidad de éxito \citep{borden2022cancer}. Incluso varias compañías como BioNTech, Genocea Biosciences, Neon Therapeutics y Gritstone Oncology realizan investigación y ofrecen el servicio de generar vacunas personalizadas a pacientes de cáncer.

Según lo mencionado anteriormente, la detección de neo antígenos es un factor clave en el desarrollo de vacunas personalizadas. Para realizar dicha detección es necesario un proceso que involucra varias etapas desde el secuenciameinto de \textit{Deoxyribonucleic Acid} (DNA), detección de mutaciones, generación de péptidos (posibles neo antígenos), predicción del enlace péptido - \textit{Major Histocompatibility Complex} (pMHC) y luego la predicción del enlace pMHC con \textit{T-cell Receptor} (pMHC-TCR). De este proceso, las dos ultimas fases son muy importantes porque de ellas depende que péptido puede calificar como posible neo antígeno. 


% falta definir bien el problema indicando las entrada y salidas

Existen varias propuestas y la mayoria se basan en la aplicación de \textit{deep learning}, sin embargo solo el 3\% de neo antígenos detectados logran activar el sistema inmune \citep{de2020neoantigen}. Además, los métodos de mayor acierto en la predicción del enlace pMHC-II \footnote{Existen tres tipos de proteínas MHC: MHC-I, MHC-II y MHC-III, según el tipo de célula.} solo llega a un 60\% de acierto.



%% ------------------------------------------------------------------- %%
\section{Objetivos}
\label{sec:objetivos}


Entre los objetivos específicos del trabajo, se tienen los seguientes:

\begin{enumerate}[(a)]
\item Proponer  ......................;
\item Proponer  ......................;
\item Proponer  ......................;
\end{enumerate}

%% ------------------------------------------------------------------- %%
\section{Contribuciones}
\label{sec:contribuciones}
Las principales contribuciones de este trabajo son:

\begin{enumerate}[(a)]
	\item Una .........................;
	\item Una .........................;
	\item Una .........................;
	
\end{enumerate}

%% ------------------------------------------------------------------- %%
\section{Organización del Trabajo}
\label{sec:organizaciondeltrabajo}
En el Capítulo~\ref{cap:marcoteorico} se presentan los conceptos básicos ....

en el Capítulo~\ref{cap:estadodelarte} se describen los trabajos relacionados a la presente tesis.

..............................

Finalmente, en el Capítulo~\ref{cap:conclusiones} son expuestos las conclusiones del presente trabajo así como 
tambien las direcciones para continuar con el mismo en la sección de trabajos futuros.

%Em anexos está uma exemplificação da especificação da arquitetura, 
%neste caso é descrita a especificação da arquitetura 
%DAS-3 (Apêndice \ref{ape:especificacoes}).


%%%%%
